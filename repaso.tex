\documentclass{article}

\usepackage{amsmath}
\usepackage{amssymb}

\begin{document}
  \section{Punto 1}
    Sea $K\subseteq \mathbb{R}$ compacto no vacío y sea $(f_n)_n$ una sucesión de 
    funciones continuas uniformemente convergentes en $K$. Demostrar que $(f_n)_n$ 
    es equicontinua.

    Como cada $f_n$ es continua y $K$ es un compacto, entonces, $f_n$ es uniformemente continua,
    por lo tanto, para cada $n$, y dado $\epsilon>0$ existe $\delta_n > 0$ tal que 
      \[
        \left| f_n(x) - f_n(y)\right| < \epsilon \text{ si } \left|x - y\right| < \delta_n
      \]

    Para $\epsilon > 0$, debe existir un natural $N$ tal que si $n \ge N$, 
      \[
        \left| f_n(x) - f_N(x) \right| < \epsilon
      \]

    Luego,
      \begin{equation}
        \begin{aligned}
          \left|f_n(x) - f_n(y)\right| &\le \left|f_n(x) - f_N(x)\right| + \left|f_N(x) - f_n(y)\right| \\
          &\le \left|f_n(x) - f_N(x)\right| + \left|f_N(x) - f_N(y)\right| + \left|f_N(y) - f_n(y)\right|
        \end{aligned}
      \end{equation}

      A partir de ahora, consideramos $\delta = \min \delta_i, i = 1, ..., N-1$, 
      entonces la condición de equicontinuidad se verfica para las $N-1$ primeras funciones, y para 
      $n\ge N$, retomamos la inecuación demostrada anteriormente, entonces, si $|x-y| < \delta$, 
      entonces $\left|f_N(x) - f_N(y)\right|<\epsilon$, luego,

      \[
        \left|f_n(x) - f_N(x)\right| + \left|f_N(x) - f_N(y)\right| + \left|f_N(y) - f_n(y)\right| < 3\epsilon
      \]
    Concluyendo así que $(f_n)_n$ es equicontinua.

  \section{Punto 2}
    Considerar la sucesión $(f_n)_n$ de funciones Riemann integrables y uniformemente acotadas en 
    $[a, b]$ y sea
      \[F_n(x) = \int_a^x f_n(t)\, dt, x\in[a, b]\]
    Demuestre que existe una subsucesión de esta sucesión que converge uniformemente en $[a, b]$.

    Primero vemos que $\left|F_n(x)\right| \le \int_a^x \left|f_n(t)\right| dt \le M(b-a)$, es decir,
    la sucesión es acotada puntualmente. Ahora consideramos lo siguiente:
      \[
        \left|F_n(x) - F_n(y)\right| = \left|\int_y^x f(t) dt\right| \le M|x-y|
      \]

    Es decir, cada $F_n$ es de tipo Lipchitz, y como el $M$ vale para toda la sucesión $(f_n)_n$, 
    concluimos que para $\epsilon > 0$ basta tomar $\delta = \frac{\epsilon}{M}$.


    Con lo anterior, sabemos que la sucesión es equicontinua y acotada, de modo que por el teorema 
    de completitud de funciones visto en clase, concluimos que existe una subsucesión de $F_n$ que 
    converge uniformemente en $[a, b]$.

  \section{Punto 3}
    Sea $(f_n)_n$ una sucesión equicontinua y puntualmente convergente en un compacto $K\subseteq \mathbb{R}$
    no vacío. Entonces $(f_n)_n$ converge uniformemente.


    Sea $\epsilon > 0$, entonces existe $\delta > 0$ que cumple la condición de equicontinuidad, además
    de que para cada $x$ existe un $N_x$ que satisface la condición de convergencia puntual. 
    Tomamos la familia $B_x = (x - \delta, x + \delta)$ es un cubrimiento por abiertos de $K$, por 
    lo tanto existe una cantidad finita de elementos $x_i, i = 1, ..., n$ tal que $B_{x_i}$ sigue 
    cubriendo a $K$.

    Elegimos $N= \max N_{x_i}, i = 1, ..., n$, entonces, para $m\ge N$ tenemos que

      \begin{equation}
        \begin{aligned}
          |f_m(x) - f(x)| &\le |f_m(x) - f_m(x_i)| + |f_m(x_i) - f(x)|       
        \end{aligned}
      \end{equation}

    Siempre que $|x - x_i| < \delta, \, |f_m(x) - f_m(x_i)| < \epsilon$ y como $N > N_{x_i}$, entonces
    $|f_m(x_i) - f(x)| < \epsilon$, luego, 

    \[
        |f_m(x) - f(x)| < 2\epsilon
    \]


  \section{Punto 4}
    Determinar completamente las series de Fourier de las siguientes funciones y dado el caso explicar
    en qué sentido se tiene la convergencia hacia $f(x)$. Verificar la identidad de Parseval.
    \subsection{Punto a}
      \[f(x) = 3\cos^3(x), \, |x| \le \pi\]

      Primero que todo, $\cos^3(x) = \frac{1}{2}\cos(x)(1 + \cos(2x)) = \frac{1}{2}(\cos(x)+\cos(x)\cos(2x))$,
      y recordando que $\cos(x)\cos(y) = \frac{1}{2}(\cos(x+y) + \cos(x-y))$, llegamos a 

      \begin{equation}
        \begin{aligned}
          \frac{1}{2}(\cos(x)+\cos(x)\cos(2x)) &= 
          \frac{1}{2}(\cos(x) + \frac{1}{2}(\cos(3x) + \cos(x))) \\
          &= \frac{1}{4}(3\cos(x) + \cos(3x)) \\
          &= \frac{3}{4}\cos(x) + \frac{1}{4}\cos(3x)
        \end{aligned}
      \end{equation}

      Es decir, $f(x) = \frac{9}{4}\cos(x) + \frac{3}{4}\cos(3x)$. Por la paridad de la función
      $\cos(\cdot)$, concluimos que $f$ es una función par, por lo tanto los coeficientes $b_n$ son nulos.
      

      Ahora determinemos una forma para $f(x)\cos(nx)$.

      \begin{equation}
        \begin{aligned}
          \frac{9}{4}\cos(x)\cos(nx) + \frac{1}{4}\cos(3x)\cos(nx) &=\\
          \frac{9}{8}\cos((n+1)x) + \frac{9}{8}\cos((n-1)x) + \frac{1}{8}\cos((n-3)x) + \frac{1}{8}\cos((n+3)x)
        \end{aligned}
      \end{equation}

      Analizando los valores cuando $n = 1$ y cuando $n = 3$ se obtine un resultado concluyente, ya
      que el segundo y tercer término no resultan en $0$ cuando se efectua la integral, entonces 
      los hayamos explícitamente: 

      \[
        a_1 = \frac{1}{\pi}\int_{-\pi}^{\pi} 
        \left(\frac{9}{8}\cos(2x) + \frac{9}{8} + \frac{1}{8}\cos(2x) + \frac{1}{8}\cos(4x)\right)\, dx 
        = \frac{9}{8\pi}\cdot 2\pi = \frac{9}{4}
      \]

      \[
        a_3 = \frac{1}{\pi}\int_{-\pi}^{\pi} \frac{1}{8}\, dx = \frac{1}{4}
      \]

      Por inspección, los demás términos son nulos. La serie de Fourier resulta en 
      \[
        f(x) \sim \frac{9}{4}\cos(x) + \frac{3}{4}\cos(3x)
      \]
      Que resulta siendo una de las formas que dimos arriba.


      Sabemos que la función $f$ es continua y periodica, con periodo de $2\pi$, y además su 
      derivada, $f'(x) = -9\cos^2(x)\sin(x)$ es una función Riemann integrable, entonces
      su serie de Fourier converge uniforme y absolutamente a $f$ (caso trivial).
      

      Ahora vemos que $\sum_{n= 0}^\infty |c_n|^2 = \frac{81}{16} + \frac{9}{16} = \frac{45}{8}$, 
      y por otro lado $||f||^2 = \int_{-\pi}^{\pi} |f(x)|^2\, dx$  que resulta en 

      \begin{equation}
        \frac{1}{\pi}\int_{-\pi}^{\pi} \left(\frac{81}{16}\cos^2(x) + \frac{27}{8}\cos(x)\cos(3x)+ \frac{9}{16}\cos^2(3x)\right)\, dx
      \end{equation}

      el primer sumando es $\frac{81}{16}$, el segundo es $0$ y el tercero $\frac{9}{16}$, que son los
      mismos coeficientes presentados antes, por lo tanto sus sumas coinciden.
    \subsection{Punto b}
      \[f(x) = \sqrt{\pi}, \text{ si } |x|\le \frac{1}{2},\, 0 \text{ si } \frac{1}{2}< |x|\le \pi\]


      Esta función es una función par, entonces los coeficientes  $b_n$ son nulos.


      Hallamos los coeficientes $a_n$: 

      \[
        a_n = \frac{1}{\pi} \int_{-\frac{1}{2}}^{\frac{1}{2}} \sqrt{\pi}\cos(nx)\, dx  
           = \frac{1}{\sqrt{\pi}n} \sin(nx)|_{-\frac{1}{2}}^{\frac{1}{2}}
           = \frac{2}{\sqrt{\pi}n} \sin\left(\frac{n}{2}\right)
      \]

      Y es fácil ver que $a_0 = \frac{1}{\sqrt{\pi}}$

      Entonces 

      \[
        f(x) \sim \frac{1}{2\sqrt{\pi}}  + \sum_{n=1}^\infty \frac{2}{\sqrt{\pi}n}\sin\left(\frac{n}{2}\right)\cos(nx)
      \]

      Por el criterio de Jordan, como $f$ es variación acotada sobre todo intervalo de la forma
      $[x - \delta, x + \delta],\,  0< \delta < \pi$, entonces la serie de Fourier converge hacia 
      $s(x) = \lim_{t\to 0^+}\frac{1}{2}(f(x+t) + f(x-t))$, entonces veamos qué apariencia tiene $s(x)$
      

      Si $|x| < \frac{1}{2}$, entonces $s(x) = \sqrt{\pi}$, y si $\frac{1}{2} < |x| < \pi$, $s(x)= 0$.
      Por último, si $x = \frac{1}{2}$, $s(x) = \lim_{t\to 0^+} \frac{1}{2}(\sqrt{\pi} + 0) = \frac{\sqrt{\pi}}{2}$,
      si $x = -\frac{1}{2}$, tenemos un caso análogo, entonces

      \begin{equation}
        s(x) = 
        \begin{cases}
          \sqrt{\pi}, &\text{ si } |x| < \frac{1}{2} \\
          0, &\text{ si } \frac{1}{2} < |x| \le \pi \\
          \frac{\sqrt{\pi}}{2}, &\text{ si } |x| = \frac{1}{2}
        \end{cases}
      \end{equation}

      Ahora verifiquemos la identidad de Parseval. Para ello verifiquemos el valor de $||f||^2$:

      \[
        ||f||^2 = \frac{1}{\pi}\int_{-\pi}^{\pi} f^2(x)\, dx = \int_{-\frac{1}{2}}^{\frac{1}{2}} dx = 1
      \]

      Y como $||f||^2 < \infty$, entonces 

      \[
        \frac{1}{2\pi} + \frac{4}{\pi^2}\sum_{n=1}^\infty \frac{1}{n^2}\sin^2\left(\frac{n}{2}\right) = 1
      \]

    \subsection{Punto c}
      \[f(x) = x^2, \text{ si } |x| \le \pi\]

      De nuevo, esta es una función par, entonces los coeficientes $b_n$ son $0$. Los coeficientes 
      $a_n$ vienen dados por:
        \begin{equation}
          \begin{aligned}
            a_n &= \frac{1}{\pi}\int_{-\pi}^{\pi} x^2\cos(nx)\, dx \\
               &= -\frac{2}{n\pi} \int_{-\pi}^{\pi} \sin(nx)x\, dx \\
               &= -\frac{2}{n\pi} (-\frac{2}{n}\cos(n\pi)\pi) \\
               &= \frac{4}{n^2} (-1)^n
          \end{aligned}
        \end{equation}

      y es sencillo verificar que $a_0 = \frac{2}{3}\pi^2$, concluyendo que 
      \[
        f(x) \sim \frac{\pi^2}{3} + 4\sum_{n=1}^\infty \frac{(-1)^n}{n^2} \cos(nx)
      \]  

      De nuevo usamos el lema de Jordan, como $x^2$ es de variación acotada en el intervalo 
      $[x-\delta, x+\delta], 0 < \delta < \pi$, para todo $x$, entonces la serie converge hacia
      $s(x) = \lim_{t\to 0} \frac{f(x+t) + f(x-t)}{2} = x^2$ ya que $f$ es continua. 

      De nuevo, como $||f||^2 < \infty$ entonces se cumple la identidad de Parseval, y como 
      \[
        \int_{-\pi}^{\pi} x^4\, dx = \frac{2}{5}\pi^5
      \]

      Entonces 
        \[
          \frac{2}{5}\pi^4 = \frac{2\pi^4}{9} + \sum_{n=1}^\infty \frac{16}{n^4}
        \]

  \section{Punto 5}
    Sea $f(x) = \pi - x, 0 < x \le \pi$. Determine los coeficientes de Fourier de la extención
    impar a $[-\pi, \pi]$.

    $f$ estaría determinada como 
    \[
      f(x) =
      \begin{cases}
        \pi - x, \text{ si } 0 < x\le \pi \\
        0, \text{ si } x= 0 \\
        -(x+\pi), \text{ si } -\pi \le x < 0
      \end{cases}
    \]
    Como extendimos a $f$ en su forma impar, entonces los coeficientes $a_n$ son $0$. Para determinar 
    los coeficientes $b_n$ resolvermos:

    \[
      b_n = \frac{1}{\pi} \int_{-\pi}^{\pi} f(x)\sin(nx) \, dx = 
          \frac{-1}{\pi}\int_{-\pi}^0 (x+\pi)\sin(nx)\, dx + \frac{1}{\pi}\int_0^\pi (\pi - x)\sin(nx)\,dx
    \]
    
    Empezamos resolviendo la primera integral:
      \[
        -\frac{1}{\pi}\int_{-\pi}^0 (x+\pi)\sin(nx)\, dx = 
        -\frac{1}{\pi}\left(-\frac{\pi}{n} + \frac{1}{n}\int_{-\pi}^0 \cos(nx)\, dx\right) = \frac{1}{n}
      \]

    Y la segunda resulta en:
    \[
      \frac{1}{\pi} \int_0^\pi (\pi-x)\sin(nx)\,dx = 
      \frac{1}{\pi}\left(\frac{\pi}{n} + \frac{1}{n}\int_0^\pi \cos(nx)\,dx\right) =
      \frac{1}{n}
    \]

    Es decir, $b_n = \frac{2}{n}$ y su serie de Fourier asociada es:
      \[
        f(x) \sim 2\sum_{n=1}^\infty \frac{\sin(nx)}{n}
      \]

    Ahora análizamos la convergencia de esta serie hacia la función original. De nuevo, usamos el 
    lema de Jordan; como $f$ es de variación acotada en cada intervalo de la forma $[x - \delta, x+\delta]$,
    para $\delta < \pi$, entonces la serie coverge a $s(x) = \lim_{t\to0^+} \frac{1}{2}(f(x+t)+f(x-t))$.
    Vemos que cuando $x > 0$ o $x < 0$, $s(x) = f(x)$ y cuando $x = 0$, $s(x) = 0$, es decir,
    $s(x) = f(x)$. Ahora determinemos el valor de $\frac{1}{\pi}\int_{-\pi}^\pi f(x)\, dx$.

    \begin{equation}
      \begin{aligned}
        \frac{1}{\pi}\int_{-\pi}^\pi f^2(x)\, dx &= \frac{1}{\pi}\int_{-\pi}^0 (\pi+x)^2\,dx + \frac{1}{\pi}\int_0^\pi (\pi -x)^2\, dx \\
        &= \frac{1}{\pi} \frac{(\pi + x)^3}{3}\left|_{-\pi}^0\right. - \frac{1}{\pi} \frac{(\pi -x)^3}{3} |_0^\pi \\
        &= \frac{2}{3}\pi^2
      \end{aligned}
    \end{equation}

    Entonces $\frac{2}{3}\pi^2 = \sum_{n=1}^\infty \frac{4}{n^2}$, siendo esta la identidad de 
    Parseval. De esto último se concluye el problema de Basilea
    \[\frac{\pi^2}{6} = \sum_{n=1}^\infty \frac{1}{n^2}\]

  \section{Punto 6}
    Sea $f(x) = \sin(x), \text{ si } x\in[0, \pi]$
    \subsection{Punto a}
      Determine una serie de cosenos de $f$ en $[-\pi, \pi]$ y describir los tipos de convergencia
      hacia $f$ en $[0, \pi]$.

      Extendemos a $f$ como una función par en el intervalo $[-\pi, \pi]$ para que los coeficientes 
      $a_n$ no se anulen. 

      \[
        a_n = \frac{1}{\pi} \int_{-\pi}^\pi f(x)\cos(nx)\, dx =
        -\frac{1}{\pi}\int_{-\pi}^0 \sin(x)\cos(nx)\, dx + \frac{1}{\pi}\int_0^\pi \sin(x)\cos(nx)\,dx
      \]

      Determinemos el valor de $\int \sin(x) \cos(nx)\, dx$. Si $n\ge 2$, entonces
      
      \begin{equation}
        \begin{aligned}
          \int \sin(x)\cos(nx)\, dx &= -\cos(x)\cos(nx) - n\int \cos(x)\sin(nx)\,dx \\
          &= -\cos(x)\cos(nx) - n\left(\sin(x)\sin(nx) - n\int \sin(x)\cos(nx)\, dx\right) \\
          \int \sin(x)\cos(nx)\, dx &= \frac{1}{n^2 - 1}\left(\cos(x)\cos(nx) + n\sin(x)\sin(nx)\right) 
        \end{aligned}
      \end{equation}
      
      Con esto podemos determinar los siguientes valores cuando $n > 1$
      \[
        \int_{-\pi}^0 \sin(x)\cos(nx)\, dx = \frac{1}{n^2 - 1}(1 + (-1)^n)
      \]
      \[
        \int_0^\pi \sin(x)\cos(nx)\, dx = \frac{1}{n^2 - 1}((-1)^{n+1} - 1)
      \]

      Entonces $a_n = \frac{2}{\pi(n^2 - 1)}((-1)^{n+1} - 1)$, si $n>1$. Si $n = 1$,
      $\int \sin(x)\cos(x)\, dx = \frac{1}{2}\sin^2(x)$, entonces $a_1 = 0$, y 
      $a_0 = \frac{1}{\pi}\left(-\int_{-\pi}^0 \sin(x)\, dx + \int_0^\pi \sin(x)\, dx\right)$, por lo
      tanto, $a_0 = \frac{4}{\pi}$, con todo esto en mente, determinamos 
      \[
        f(x)\sim \frac{2}{\pi} + \frac{2}{\pi}\sum_{n=2}^\infty \frac{(-1)^{n+1} - 1}{n^2 - 1} \cos(nx)
      \]

      Analizando el término $((-1)^{n+1} - 1)$, es $0$ cuando $n$ es impar y $-2$ cuando es par, entonces
      el término $a_n = \frac{-4}{\pi(2n-1)(2n+1)}, n>1$, luego, la serie se puede escribir como
      \[
        f(x) \sim \frac{2}{\pi} - \frac{4}{\pi}\sum_{n = 1}^\infty \frac{\cos(2nx)}{(2n-1)(2n+1)}
      \]
    \subsection{Punto b}
      Ahora determinemos la identidad de Parseval, para ello, hallemos el valor de 
      $\frac{1}{\pi}\int_{-\pi}^\pi f^2(x)\, dx$

      \[
        \frac{1}{\pi}\int_{-\pi}^\pi f^2(x)\, dx = \frac{2}{\pi} \int_0^\pi \sin^2(x)\, dx =
        \frac{1}{\pi}\left(\pi - \int_0^\pi \cos(2x) \right) = 1
      \]

      Es decir que 
      \[
        1 = \frac{8}{\pi^2} + \frac{16}{\pi^2}\sum_{n=1}^\infty \frac{1}{(2n-1)^2 (2n+1)^2}
      \]

      Despejando la serie se concluye que 
        \[
        \sum_{n=1}^\infty \frac{1}{(2n-1)^2 (2n+1)^2} = \frac{\pi^2}{16} - \frac{1}{2}
      \]

  \section{Punto 7}
    Sea $f$ una función Riemann integrable $T$-periódica, $T>0$. Muestre que $F(x) = \int_0^x f(t)\, dt$
    es $T$-periódica sii $\int_0^T f(t)\, dt = 0$.

    $F(x+T) = F(x) \iff \int_0^{x+T} f(t)\, dt = \int_0^x f(t)\, dt \iff \int_0^x f(t)\, dt + \int_x^{x+T} f(t)\, dt = \int_0^x f(t)\, dt
    \iff \int_x^{x+T} f(t)\, dt = 0 \iff \int_0^T f(t)\, dt = 0$ 

  \section{Punto 8}
    Muestre que $\sum_{n=1}^\infty \frac{1}{\sqrt{n}}\sin(nx)$ no puede ser la serie de Fourier 
    de ninguna función.

    Si fuera la serie de Fourier de alguna función, se cumple que $\sum_{n=1}^\infty a_n^2 +b_n^2 < \infty$,
    en esa serie se hace notar que $a_n = 0$ y $b_n = \frac{1}{\sqrt{n}}$, pero esto es una contradición
    porque sabemos que $\sum_{n = 1}^\infty \frac{1}{n}$ diverge.

  \section{Punto 9}
    Muestre que $\sum_{n=1}^\infty \frac{b_n}{n} = \infty$, entonces $\sum_{n=1}^\infty b_n \cos(nx)$
    no puede ser la serie de Fourier de de una función continua y periodica.

    Por la desigualdad de Cauchy-Schwarz sabemos que
    \[
      \left|\sum_{n=1}^\infty \frac{b_n}{n}\right| \le \left(\sum_{n=1}^\infty b_n^2\right)^{\frac{1}{2}}
      \left(\sum_{n=1}^\infty \frac{1}{n^2}\right)^{\frac{1}{2}}
    \]

    Si $\sum_{n=1}^\infty b_n \cos(nx)$ fuera la serie de Fourier de alguna función, sabemos que 
    $\sum_{n=1}^\infty b_n^2 < \infty$, lo que contradice que $\sum_{n=1}^\infty \frac{b_n}{n} = \infty$.
\end{document}
