\documentclass{article}
\usepackage{pdfpages}

\begin{document}
  \includepdf[pages={1-2}]{anexos/cedula.pdf}
  \includepdf[pages={1-2}]{anexos/TIUN.pdf}

  \begin{center}
    {\Large ¿De qué manera puedo contribuir a los estudiantes PEAMA Sumapaz
    para que desarrollen habilidades de regulación de sus procesos de aprendizaje en el
    marco de la asignatura apoyada?}
  \end{center}

  \noindent En lo personal, siempre he convivido con comunidades rurales: he conocido y practicado sus costumbres.
  Mi madre perteneció a la comunidad campesina de Boyacá y, por eso mismo, me siento identificado como un campesino
  más. Esto me ha permitido visibilizar algunas problemáticas frecuentes en los procesos formativos de los colegios
  rurales, siendo una de las principales el fortalecimiento del estudio autónomo.
  \newline
  \newline
  \noindent Como tutor, me interesaré tanto por el proceso formativo de estos estudiantes aportándoles mis herramientas
  y conocimientos del área de cálculo--- como por aspectos personales que influyen directamente en su aprendizaje
  (motivación, hábitos de estudio, confianza y manejo del tiempo). Mi objetivo será acompañarlos para que planifiquen
  su estudio, monitoreen su avance y evalúen sus estrategias, de manera que cada vez dependan menos de la guía externa
  y ganen autonomía en su proceso de aprendizaje.
  \newline
  \newline
  \noindent Para incentivar la autonomía, cada vez que me reúna con los tutorados dejaré ejercicios que promuevan el
  pensamiento crítico. Serán actividades en las que deban integrar lo aprendido, justificar procedimientos y explicar
  sus decisiones (por qué usan tal método, qué significa el resultado y cómo pueden verificarlo). Con esto busco que
  reconozcan qué tipo de práctica realmente fortalece su aprendizaje y que, con el tiempo, puedan diseñar su propio
  plan de estudio sin depender totalmente del tutor.
  \newline\newline
  \noindent Además, durante las tutorías trabajaremos con una estructura sencilla de autorregulación:
  \textit{(i)} antes de empezar, definir una meta clara para la sesión (por ejemplo, ``comprender el significado de la derivada y resolver tres problemas aplicados'');
  \textit{(ii)} durante el trabajo, monitorear el avance con preguntas guía (``¿entiendo cada paso?, ¿qué parte me cuesta?, ¿qué error se repite?'');
  y \textit{(iii)} al final, evaluar lo logrado y acordar un plan de mejora (qué repasar, qué ejercicios hacer y cómo saber si se dominó el tema).
  Este ciclo les permite transformar el estudio en un proceso consciente y ordenado, no en una tarea improvisada.
  \newline\newline
  \noindent Finalmente, procuraré que el acompañamiento sea pertinente para su contexto: reconocer sus tiempos, su acceso
  a recursos y su experiencia previa, para proponer estrategias realistas (rutinas cortas pero constantes, uso de cuadernos
  de errores, y hábitos de estudio sostenibles). De esta manera, mi contribución no se limitará a resolver ejercicios de
  cálculo, sino a fortalecer habilidades que les servirán en cualquier asignatura: planear, persistir, corregir y aprender
  con autonomía.
\end{document}
