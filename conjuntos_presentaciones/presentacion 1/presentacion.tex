\documentclass{beamer}

\begin{document}

  \begin{frame}{Definiciones básicas}
    \begin{center}
      {\Large ¿Qué es lo más básico de la teoría?}
    \end{center}

    La lógica hace parte de la conciencia humana.
    
    \begin{center}
      {\Large Ejemplos de proposiciones}
    \end{center}

    \begin{itemize}
      \item En Bogotá llueve todos los días.
      \item Las vacas vuelan.
      \item Todo natural es par.
    \end{itemize}

     \begin{center}
      {\Large Ejemplos de frases que no son proposiciones}
    \end{center}

    \begin{itemize}
      \item ¿Cómo está la situación de la unal?
      \item !Que semestre tan complicado¡.
      \item ¿Todo ordinal es un cardinal?
    \end{itemize}
    
    De esta motivación vemos que toda proposición debe terner un valor de verdad único e invariante.
  \end{frame}

  \begin{frame}{Operaciones entre proposiciones}
    Es completamente natural querer unir frases y obtener nuevas a partir de frases ya existentes.

    Esto motiva que toda próxima operación definida se fundamenta única y exclusivamente en las ideas lógicas.

    Si tenemos dos proposiciones $p$ y $q$ definimos la oración $p \land q$ que en lenguaje natural se refiere a unir 
    dos oraciones con una "y", es llamada conjunción.

    Similarmente se define la oración $p \lor q$ que en lenguaje natural se refiere a unir dos frases con una "o".

    Por último, se define la negación, $\neg p$.

    El valor de verdad de la proposión depende de los valores de verdad de las proposiciones iniciales.
  \end{frame} 
  
  \begin{frame}{Tablas de verdad}
    \begin{columns}[t]

        \begin{column}{0.32\textwidth}
           \begin{tabular}{c|c|c}
              $p$ & $q$ & $p \land q$ \\ \hline
              V   &  V  & V \\ \hline
              V   &  F  & F \\ \hline 
              F   & V   & F \\ \hline 
              F   & F   & F \\ \hline
          \end{tabular}
        \end{column}

        \begin{column}{0.32\textwidth}
           \begin{tabular}{c|c|c}
              $p$ & $q$ & $p \lor q$ \\ \hline
              V   &  V  & V \\ \hline
              V   &  F  & V \\ \hline 
              F   & V   & V \\ \hline 
              F   & F   & F \\ \hline
          \end{tabular}
        \end{column}
        
         \begin{column}{0.32\textwidth}
           \begin{tabular}{c|c}
              $p$ & $\neg p$ \\ \hline
              V  & F \\ \hline
              F  & V \\ \hline
          \end{tabular}
        \end{column}

    \end{columns}
  \end{frame}

  \begin{frame}{Proposiciones complejas usadas en la teoría}
    Tomemos dos proposiciones $t, p$ y $q$, definimos las siguientes proposiciones complejas 
    \begin{itemize}
      \item $p$ entonces $q$, $p \rightarrow q := \neg p \lor q$.
      \item $p$ si y sólo si $q$, $p \iff q := (p \rightarrow q) \land (q \rightarrow p)$.
      \item $p$ xor $q$, $p \oplus q := \neg (p \iff q)$.

      Mas adelante se harán análogias sobre contenencia de conjuntos basados en las dos primeras 
      proposiciones complejas. Una proposición compleja es una tautologia siempre que sin importar
      el valor de verdad de sus proposiciones mas simples, se llega a una conclusión verdadera.
    \end{itemize}

    {\large Leyes de Morgan}

    \begin{itemize}
      \item $\neg (p\land q) = \neg p \lor \neg q$.
      \item $\neg (p\lor q) = \neg p \land \neg q$.
      \item $t\land (p\lor q) = (t\land p) \lor (t\land q)$.
      \item $t\lor (p\land q) = (t\lor p) \land (t\lor q)$.
    \end{itemize}
    La equivalencia de las leyes de Morgan se hace por medio de tablas de verdad.
    \end{frame}

    \begin{frame}{Sección 1 de ejercicios}
      \begin{itemize}
        \item Haga la tabla de verdad de cada proposición compleja, identificando aspectos
              claves que tiene una con respecto a las demás.

        \item Establesca la equivalencia de las leyes de Morgan.

        \item Realice la tabla de verdad de la proposición $(p \land q) \lor ((p \iff q) \land (p \oplus q))$. Proposiciones como 
              esta son llamadas contradicciones. ¿Qué proposiciones básicas deben ser removidas para que no sea una contradicción?
      \end{itemize}
    \end{frame}

    \begin{frame}{Álgebra booleana}
      \begin{center}
        {\Large Propiedades básicas de las operaciones básicas}
      \end{center}
      \begin{itemize}
        \item $p\land p = p\lor p = p$.
        \item $\neg (\neg p) = p$.
        \item $p \land q = q \land p$.
        \item $p \lor q = q \lor p$.
        \item $p \land (q \land t) = (p\land q) \land t$.
        \item $p \lor (q \lor t) = (p\lor q) \lor t$.
      \end{itemize}
        Cualquier operación lógica puede ser disminuida a operaciones con $\land, \lor$ y $\neg$, pero una herramienta mas
        útil es, cuando se pueda, escribirlas en términos de proposiciones complejas, ya que son herramientas de simplificación.

      {\large Ejemplo}

        $p \rightarrow q \equiv \neg p \lor q \equiv \neg p \lor \neg (\neg q) \equiv \neg (\neg q) \lor \neg p \equiv \neg q \rightarrow \neg p$.

        Esta propiedad recibe el nombre de contraposición.
    \end{frame}

    \begin{frame}{Reglas de inferencia}
      A partir de ahora se quiere concluir proposiciones a partir de otras que ya sabemos que son verdad,
      a continuación detallo el proceso. Las proposiciones inciales reciben el nombre de premisas y se consideran verdaderas.

      {\large Ejemplos}
        
        \((p \rightarrow q) \land p \Rightarrow q\)

        Para esto, podemos que ver que $p \rightarrow q$ es verdad y que $p$ es verdad. Como $p$ es verdad, si $q$ fuera falso
        entonces se llega a que $p \rightarrow q$ es falso, lo cual es una contradicción, llegando a que $q$ es verdad.
        

        \((p \rightarrow q) \land \neg q \Rightarrow \neg p\)

        De nuevo, como $\neg q$ es verdad, entonces $q$ es falso, entonces $p$ no puede ser verdad, por lo tanto $\neg p$ es verdad.

        \(p\land q \Rightarrow p\)

        Esta ya la usamos antes, ya que es demasiado lógica, se llama simplificación.
    \end{frame}

    \begin{frame}
        \(q \Rightarrow p\lor q\)

        \((p\lor q)\land \neg p \Rightarrow q\)

        Como $p$ es falso, concluimos que $q$ es verdad.

        \(p \iff q \Rightarrow p\rightarrow q\)

        \((p\rightarrow q) \land (q \rightarrow t) \Rightarrow p\rightarrow t \)
          
          Esta es una prueba por casos. Si $p$ es falso, entonces se concluye inmediatamente. Si $p$ es verdadero,
          entonces $q$ debe ser verdadero, porque de lo contrario se llega a una contradicción, y por el mismo argumento,
          concluimos que $t$ debe ser verdadero, llegando a que $p\rightarrow t$ es verdad.


        Otra forma de establecer estas implicaciones es hacer la tabla de verdad de la proposición $[premisas] \rightarrow consecuente$
        y verificar que es una tautología.
    \end{frame}
\end{document}
