\documentclass{beamer}
\newcommand{\pair}[2]{(#1,\, #2)}

\begin{document}
  \begin{frame}{Parejas ordenadas}
    \textbf{Definicion:} Se dice que un conjunto $A$ es una pareja ordenada si existen conjuntos 
    $B,\, C$ tales que $A = \{\{B\},\, \{B,\, C\}\}$, en cuyo caso se denota $A = (B,\, C)$.
  \end{frame}

  \begin{frame}{Parejas ordenadas}
    \textbf{Definicion:} Se dice que un conjunto $A$ es una pareja ordenada si existen conjuntos 
    $B,\, C$ tales que $A = \{\{B\},\, \{B,\, C\}\}$, en cuyo caso se denota $A = (B,\, C)$.
    
    \textbf{Teorema 11:} Dados $a,\, b$ conjuntos, existe la pareja ordenada $(a,\, b)$.
  \end{frame}

  \begin{frame}{Parejas ordenadas}
    \textbf{Definición:} Se dice que un conjunto $A$ es una pareja ordenada si existen conjuntos 
    $B,\, C$ tales que $A = \{\{B\},\, \{B,\, C\}\}$, en cuyo caso se denota $A = (B,\, C)$.
    
    \textbf{Teorema 11:} Dados $a,\, b$ conjuntos, existe la pareja ordenada $(a,\, b)$.

    Por el axioma 4, existen $\{a\}$ y $\{a,\, b\}$, y de nuevo por este axioma, existe 
    $\{\{a\},\, \{a,\, b\}\}$.
  \end{frame}

  \begin{frame}{Parejas ordenadas}
    \textbf{Definicion:} Se dice que un conjunto $A$ es una pareja ordenada si existen conjuntos 
    $B,\, C$ tales que $A = \{\{B\},\, \{B,\, C\}\}$, en cuyo caso se denota $A = (B,\, C)$.
    
    \textbf{Teorema 11:} Dados $a,\, b$ conjuntos, existe la pareja ordenada $(a,\, b)$.

    Por el axioma 4, existen $\{a\}$ y $\{a,\, b\}$, y de nuevo por este axioma, existe 
    $\{\{a\},\, \{a,\, b\}\}$.    

    \textbf{Teorema 12:} Sean $a,\, b$ conjuntos, entonces $(a,\, b) \neq (b,\, a)$.
  \end{frame}

  \begin{frame}
    \textbf{Teorema 13:} Sean $a,\, b,\, c$ y $d$ conjuntos. $(a,\, b) = (c,\, d)$
    si y sólo si, $a = c$ y $b = d$.
  \end{frame}

  \begin{frame}
    \textbf{Teorema 13:} Sean $a,\, b,\, c$ y $d$ conjuntos. $(a,\, b) = (c,\, d)$
    si y sólo si, $a = c$ y $b = d$.

    $\Longrightarrow)$

    Consideremos que $\pair{a}{b} = \pair{c}{d} $. 
  \end{frame}

  \begin{frame}
    \textbf{Teorema 13:} Sean $a,\, b,\, c$ y $d$ conjuntos. $(a,\, b) = (c,\, d)$
    si y sólo si, $a = c$ y $b = d$.

    $\Longrightarrow)$

    Consideremos que $\pair{a}{b} = \pair{c}{d} $. 
    \begin{itemize}
      \item Si $c = d$.
    \end{itemize}
  \end{frame}

  \begin{frame}
    \textbf{Teorema 13:} Sean $a,\, b,\, c$ y $d$ conjuntos. $(a,\, b) = (c,\, d)$
    si y sólo si, $a = c$ y $b = d$.

    $\Longrightarrow)$

    Consideremos que $\pair{a}{b} = \pair{c}{d} $. 
    \begin{itemize}
      \item Si $c = d$.
            
            Como $\{a\} \in \pair{c}{d} = \{\{c\}\}$, entonces $\{a\} = \{c\}$, luego $a = c$.
            Ahora, como $\{a, b\} \in \pair{c}{d} = \{\{a\}\}$, entonces $\{a, b\} = \{a\}$, luego,
            $b = a = c = d$.
    \end{itemize}
  \end{frame}

  \begin{frame}
    \textbf{Teorema 13:} Sean $a,\, b,\, c$ y $d$ conjuntos. $(a,\, b) = (c,\, d)$
    si y sólo si, $a = c$ y $b = d$.

    $\Longrightarrow)$

    Consideremos que $\pair{a}{b} = \pair{c}{d} $. 
    \begin{itemize}
      \item Si $c \neq d$.
    \end{itemize}
  \end{frame}

  \begin{frame}
    \textbf{Teorema 13:} Sean $a,\, b,\, c$ y $d$ conjuntos. $(a,\, b) = (c,\, d)$
    si y sólo si, $a = c$ y $b = d$.

    $\Longrightarrow)$

    Consideremos que $\pair{a}{b} = \pair{c}{d} $. 
    \begin{itemize}
      \item Si $c \neq d$.
            
        De nuevo, como $\{a\} \in \pair{c}{d}$, entonces $\{a\} = \{c\}$ o $\{a\} = \{c, d\}$
        y, $\{a, b\} = \{c\}$ o $\{a, b\} = \{c, d\}$.
    \end{itemize}
  \end{frame}

  \begin{frame}
    \textbf{Teorema 13:} Sean $a,\, b,\, c$ y $d$ conjuntos. $(a,\, b) = (c,\, d)$
    si y sólo si, $a = c$ y $b = d$.

    $\Longrightarrow)$

    Consideremos que $\pair{a}{b} = \pair{c}{d} $. 
    \begin{itemize}
      \item Si $c \neq d$.
            
        De nuevo, como $\{a\} \in \pair{c}{d}$, entonces $\{a\} = \{c\}$ o $\{a\} = \{c, d\}$
        y, $\{a, b\} = \{c\}$ o $\{a, b\} = \{c, d\}$.

        \begin{itemize}
          \item Si $\{a\} = \{c, d\}$.
        \end{itemize}
    \end{itemize}
  \end{frame}

  \begin{frame}
    \textbf{Teorema 13:} Sean $a,\, b,\, c$ y $d$ conjuntos. $(a,\, b) = (c,\, d)$
    si y sólo si, $a = c$ y $b = d$.

    $\Longrightarrow)$

    Consideremos que $\pair{a}{b} = \pair{c}{d} $. 
    \begin{itemize}
      \item Si $c \neq d$.
            
        De nuevo, como $\{a\} \in \pair{c}{d}$, entonces $\{a\} = \{c\}$ o $\{a\} = \{c, d\}$
        y, $\{a, b\} = \{c\}$ o $\{a, b\} = \{c, d\}$.

        \begin{itemize}
          \item Si $\{a\} = \{c, d\}$.

                Podemos ver que $d \in \{a\}$ y $c \in \{a\}$, luego $d = a = c$, lo cual es contradictorio.
        \end{itemize}
    \end{itemize}
  \end{frame}

  \begin{frame}
    \textbf{Teorema 13:} Sean $a,\, b,\, c$ y $d$ conjuntos. $(a,\, b) = (c,\, d)$
    si y sólo si, $a = c$ y $b = d$.

    $\Longrightarrow)$

    Consideremos que $\pair{a}{b} = \pair{c}{d} $. 
    \begin{itemize}
      \item Si $c \neq d$.
            
        De nuevo, como $\{a\} \in \pair{c}{d}$, entonces $\{a\} = \{c\}$ o $\{a\} = \{c, d\}$
        y, $\{a, b\} = \{c\}$ o $\{a, b\} = \{c, d\}$.

        \begin{itemize}
          \item Si $\{a\} = \{c, d\}$.

                Podemos ver que $d \in \{a\}$ y $c \in \{a\}$, luego $d = a = c$, lo cual es contradictorio.
        \end{itemize}

        De esto último, vemos que $\{a\} = \{c\}$, luego $a = c$.
    \end{itemize}
  \end{frame}

  \begin{frame}
    \textbf{Teorema 13:} Sean $a,\, b,\, c$ y $d$ conjuntos. $(a,\, b) = (c,\, d)$
    si y sólo si, $a = c$ y $b = d$.

    $\Longrightarrow)$

    Consideremos que $\pair{a}{b} = \pair{c}{d} $. 
    \begin{itemize}
      \item Si $c \neq d$.
            
        De nuevo, como $\{a\} \in \pair{c}{d}$, entonces $\{a\} = \{c\}$ o $\{a\} = \{c, d\}$
        y, $\{a, b\} = \{c\}$ o $\{a, b\} = \{c, d\}$.

        \begin{itemize}
          \item Si $\{a, b\} = \{c\}$
        \end{itemize}
    \end{itemize}
  \end{frame}

  \begin{frame}
    \textbf{Teorema 13:} Sean $a,\, b,\, c$ y $d$ conjuntos. $(a,\, b) = (c,\, d)$
    si y sólo si, $a = c$ y $b = d$.

    $\Longrightarrow)$

    Consideremos que $\pair{a}{b} = \pair{c}{d} $. 
    \begin{itemize}
      \item Si $c \neq d$.
            
        De nuevo, como $\{a\} \in \pair{c}{d}$, entonces $\{a\} = \{c\}$ o $\{a\} = \{c, d\}$
        y, $\{a, b\} = \{c\}$ o $\{a, b\} = \{c, d\}$.

        \begin{itemize}
          \item Si $\{a, b\} = \{c\}$.
      
                De lo anterior, viemos que $c = a$, y como $b \in \{c\}$, concluimos que $b = c$.
        \end{itemize}
    \end{itemize}
  \end{frame}

  \begin{frame}
    \textbf{Teorema 13:} Sean $a,\, b,\, c$ y $d$ conjuntos. $(a,\, b) = (c,\, d)$
    si y sólo si, $a = c$ y $b = d$.

    $\Longrightarrow)$

    Consideremos que $\pair{a}{b} = \pair{c}{d} $. 
    \begin{itemize}
      \item Si $c \neq d$.
            
        De nuevo, como $\{a\} \in \pair{c}{d}$, entonces $\{a\} = \{c\}$ o $\{a\} = \{c, d\}$
        y, $\{a, b\} = \{c\}$ o $\{a, b\} = \{c, d\}$.

        \begin{itemize}
          \item Si $\{a, b\} = \{c\}$.
      
                De lo anterior, viemos que $c = a$, y como $b \in \{c\}$, concluimos que $b = c$. 
                Una lectura mas detallada sugiere que $b \neq d$. Con esto último tenemos que 
        \end{itemize}
    \end{itemize}
  \end{frame}

  \begin{frame}
    \textbf{Teorema 13:} Sean $a,\, b,\, c$ y $d$ conjuntos. $(a,\, b) = (c,\, d)$
    si y sólo si, $a = c$ y $b = d$.

    $\Longrightarrow)$

    Consideremos que $\pair{a}{b} = \pair{c}{d} $. 
    \begin{itemize}
      \item Si $c \neq d$.
            
        De nuevo, como $\{a\} \in \pair{c}{d}$, entonces $\{a\} = \{c\}$ o $\{a\} = \{c, d\}$
        y, $\{a, b\} = \{c\}$ o $\{a, b\} = \{c, d\}$.

        \begin{itemize}
          \item Si $\{a, b\} = \{c\}$.
      
                De lo anterior, viemos que $c = a$, y como $b \in \{c\}$, concluimos que $b = c$. 
                Una lectura mas detallada sugiere que $b \neq d$. Con esto último tenemos que 

                \begin{equation*}
                  \begin{aligned}
                      \{a,\,b\} \neq \{c,\,d\}
                  \end{aligned}
                \end{equation*}
        \end{itemize}
    \end{itemize}
  \end{frame}

  \begin{frame}
    \textbf{Teorema 13:} Sean $a,\, b,\, c$ y $d$ conjuntos. $(a,\, b) = (c,\, d)$
    si y sólo si, $a = c$ y $b = d$.

    $\Longrightarrow)$

    Consideremos que $\pair{a}{b} = \pair{c}{d} $. 
    \begin{itemize}
      \item Si $c \neq d$.
            
        De nuevo, como $\{a\} \in \pair{c}{d}$, entonces $\{a\} = \{c\}$ o $\{a\} = \{c, d\}$
        y, $\{a, b\} = \{c\}$ o $\{a, b\} = \{c, d\}$.

        \begin{itemize}
          \item Si $\{a, b\} = \{c\}$.
      
                De lo anterior, viemos que $c = a$, y como $b \in \{c\}$, concluimos que $b = c$. 
                Una lectura mas detallada sugiere que $b \neq d$. Con esto último tenemos que 

                \begin{equation*}
                  \begin{aligned}
                    \{a,\,b\} &\neq \{c,\,d\} \\
                    \{\{a\},\, \{a,\, b\}\} &\neq \{\{c\},\, \{c,\, d\}\}
                  \end{aligned}
                \end{equation*}
        \end{itemize}
    \end{itemize}
  \end{frame}

  \begin{frame}
    \textbf{Teorema 13:} Sean $a,\, b,\, c$ y $d$ conjuntos. $(a,\, b) = (c,\, d)$
    si y sólo si, $a = c$ y $b = d$.

    $\Longrightarrow)$

    Consideremos que $\pair{a}{b} = \pair{c}{d} $. 
    \begin{itemize}
      \item Si $c \neq d$.
            
        De nuevo, como $\{a\} \in \pair{c}{d}$, entonces $\{a\} = \{c\}$ o $\{a\} = \{c, d\}$
        y, $\{a, b\} = \{c\}$ o $\{a, b\} = \{c, d\}$.

        \begin{itemize}
          \item Si $\{a, b\} = \{c\}$.
      
                De lo anterior, viemos que $c = a$, y como $b \in \{c\}$, concluimos que $b = c$. 
                Una lectura mas detallada sugiere que $b \neq d$. Con esto último tenemos que 

                \begin{equation*}
                  \begin{aligned}
                    \{a,\,b\} &\neq \{c,\,d\} \\
                    \{\{a\},\, \{a,\, b\}\} &\neq \{\{c\},\, \{c,\, d\}\} \\
                    \pair{a}{b} &\neq \pair{c}{d}
                  \end{aligned}
                \end{equation*}
        \end{itemize}
    \end{itemize}
  \end{frame}

  \begin{frame}
    \textbf{Teorema 13:} Sean $a,\, b,\, c$ y $d$ conjuntos. $(a,\, b) = (c,\, d)$
    si y sólo si, $a = c$ y $b = d$.

    $\Longrightarrow)$

    Consideremos que $\pair{a}{b} = \pair{c}{d} $. 
    \begin{itemize}
      \item Si $c \neq d$.
            
        De nuevo, como $\{a\} \in \pair{c}{d}$, entonces $\{a\} = \{c\}$ o $\{a\} = \{c, d\}$
        y, $\{a, b\} = \{c\}$ o $\{a, b\} = \{c, d\}$.

        \begin{itemize}
          \item Si $\{a, b\} = \{c\}$.
      
                De lo anterior, viemos que $c = a$, y como $b \in \{c\}$, concluimos que $b = c$. 
                Una lectura mas detallada sugiere que $b \neq d$. Con esto último tenemos que 

                \begin{equation*}
                  \begin{aligned}
                    \{a,\,b\} &\neq \{c,\,d\} \\
                    \{\{a\},\, \{a,\, b\}\} &\neq \{\{c\},\, \{c,\, d\}\} \\
                    \pair{a}{b} &\neq \pair{c}{d}
                  \end{aligned}
                \end{equation*}
        \end{itemize}

        Con esto último se concluye que $\{a, b\} = \{c, d\}$ y como $b \neq c$, entonces $b = c$.
    \end{itemize}
  \end{frame}

  \begin{frame}
    \textbf{Teorema 13:} Sean $a,\, b,\, c$ y $d$ conjuntos. $(a,\, b) = (c,\, d)$
    si y sólo si, $a = c$ y $b = d$.

    $\Longleftarrow)$

    Como $a = c$ y $b = d$, entonces $\{a\} = \{c\}$ y $\{a, b\} = \{c, d\}$, luego 
    $\pair{a}{b} = \pair{c}{d}$.
  \end{frame}

  \begin{frame}{Producto cartesiano}
    \textbf{Teorema 14:} Sean $A$ y $B$ conjuntos, entonces para todo $a\in A$ y para todo $b\in B$,
    $(a, b) \in P(P(A\cup B))$.
  \end{frame}

  \begin{frame}{Producto cartesiano}
    \textbf{Teorema 14:} Sean $A$ y $B$ conjuntos, entonces para todo $a\in A$ y para todo $b\in B$,
    $(a, b) \in P(P(A\cup B))$.

    Como $\{a\}, \{a, b\}\subseteq A\cup B$, entonces $\{a\} \in P(A\cup B)$ y $\{a, b\}\in P(A\cup B)$.
  \end{frame}

  \begin{frame}{Producto cartesiano}
    \textbf{Teorema 14:} Sean $A$ y $B$ conjuntos, entonces para todo $a\in A$ y para todo $b\in B$,
    $(a, b) \in P(P(A\cup B))$.

    Como $\{a\}, \{a, b\}\subseteq A\cup B$, entonces $\{a\} \in P(A\cup B)$ y $\{a, b\}\in P(A\cup B)$.

    De eso último, $\{\{a\}, \{a, b\}\} \subseteq P(A \cup B)$, es decir, $\pair{a}{b} \in P(P(A \cup B))$.
  \end{frame}

  \begin{frame}{Producto cartesiano}
    \textbf{Teorema 15:} Dados dos conjuntos $A$ y $B$, existe un único conjunto $C$ tal que 
    $x \in C$ si y sólo si $x = \pair{a}{b}$, para algún $a\in A$ y $b\in B$.
  \end{frame}

  \begin{frame}{Producto cartesiano}
    \textbf{Teorema 15:} Dados dos conjuntos $A$ y $B$, existe un único conjunto $C$ tal que 
    $x \in C$ si y sólo si $x = \pair{a}{b}$, para algún $a\in A$ y $b\in B$.

    Tomando el conjunto (¿?) 
      \[C=\{x: (\exists a\in A)(\exists b\in B)(x = \pair{a}{b})\}\]

    Se concluye el resultado inmediatamente.
  \end{frame}

  \begin{frame}{Producto cartesiano}
    \textbf{Teorema 15:} Dados dos conjuntos $A$ y $B$, existe un único conjunto $C$ tal que 
    $x \in C$ si y sólo si $x = \pair{a}{b}$, para algún $a\in A$ y $b\in B$.

    Tomando el conjunto (¿?) 
      \[C=\{x: (\exists a\in A)(\exists b\in B)(x = \pair{a}{b})\}\]

    Se concluye el resultado inmediatamente.

    Como se tiene existencia y unicidad, al conjunto $C$ se le denota $A \times B$ y se dice 
    que $\pair{a}{b} \in A\times B$ si y sólo si, $a\in A$ y $b\in B$.
  \end{frame}

  \begin{frame}{Ejercicio}
    \textbf{Ejercicio:} Dados dos conjuntos, $A$ y $C$, ¿es $\{B\times C: B\in A\}$ un conjunto?
  \end{frame}
\end{document}
