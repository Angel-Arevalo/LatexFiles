\documentclass{beamer}

\begin{document}
  \begin{frame}{Propiedades y cuantificadores}
    En matemáticas, un concepto importante es el de variable, por eso mismo se quiere trabajar 
    con frases que las involucren y una vez fijada una frase, determinar el valor de verdad de la frase 
    obtenida.

    Un propiedades es una "proposición" para las que ciertas variables cumplen una propiedad.
    Dado un propiedad que usa una variable se simboliza de la siguiente manera
      \[P(x) = \text{ oración que involucra} x\]

    Ejemplos de propiedades:
    \begin{itemize}
      \item $P(x) = x$ es un número par.
      \item $P(x) = x$ es divisible por $3$.
    \end{itemize}

    Claramente los ejemplos dependen de un universo predefinido.
  \end{frame}

  \begin{frame}
    Se definimos el universo como el de todas las personas, vemos que los siguientes propiedades tienen sentido:
    \begin{itemize}
      \item $P(x, y) = x$ es el hijo de $y$.
      \item $P(x) = x$ es Cubano.
    \end{itemize}


    Una vez definimos un universo y una propiedad $P(x)$, lo natural es preguntar si todos los elementos del 
    universo cumplen dicha propiedad o de que almenos uno lo haga. En el primero de estos casos, lo simbolizamos 
    $\forall x \, P(x)$ y en el segundo $\exists x \,P(x)$.

    Cuando una variable $x$ aparece en una propiedad $P(x)$ se dice que $x$ está ligada cuando se usa en 
    cuantificadores de la forma $\exists x \,P(x)$ y $\forall x \,P(x)$. Cuando una variable no está ligada 
    se dice que es una variable libre.

      \[\forall x\,(P(x) \rightarrow y)\]

    También existe un simbolo para decir que existe un único elemento del universo que cumple cierta propiedad.
      \[\exists! x \, P(x)\]
  \end{frame}
  
  \begin{frame}{Negación de cuantificadores}
    Es posible aplicar la negación a los cuantificadores, se presentan sus formas y una representación gramática.

    \begin{center}
      {\Large Negación de para todo}
    \end{center}

    $\neg \forall x\, P(x) \equiv \exists x\, \neg P(x)$; "No es cierto que para todo x en el universo P(x) sea verdad
    $\equiv$ existe almenos un elemento del universo que no cumple la propiedad".

    \begin{center}
      {\Large Negación del cuantificador existencial}
    \end{center}

    $\neg \exists x\, P(x) \equiv \forall x\, \neg P(x)$; "No existe un elemento en el universo que cumpla la propiedad
    $\equiv$ para todo elemento del universo, no se cumple la propiedad".
  \end{frame}

  \begin{frame}{Primeras equivalencias de los cuantificadores}
    \begin{itemize}
      \item $\exists x \, \exists y\, P(x, y) \equiv \exists y \, \exists x\, P(x, y)$
      \item $\forall x \, \forall y\, P(x, y) \equiv \forall y \, \forall x\, P(x, y)$
    \end{itemize}

    Para probar la primera propiedad lo hacemos de esta manera: "Como $\exists x \, \exists y\, P(x, y)$, entonces existe 
    $a$ tal que $\exists y\, P(a, y)$ y de nuevo, existe $b$ de matera que $P(a, b)$. Si $P(a, b)$ podemos afirmar que 
    $\exists y\, P(a, y)$ y asi mismo  podemos deducir que $\exists y\, \exists x \,P(x, y)$".
    
    Se presenta un esquema mucho más simple
    \begin{equation*}
      \begin{aligned}
        \exists x\, \exists y\, P(x, y) &\iff \exists y\, P(a, y), \text{ para algún } a \\
        &\iff P(a, b), \text{ para algún par } a,b \\
        &\iff \exists y, \, P(a, y), \text{ para algún } a \\
        &\iff \exists y\, \exists x\, P(x,y)
      \end{aligned}
    \end{equation*}
    Estas dos propiedad nos dice que no importa el orden en que se afirme existencia o universalidad.
  \end{frame}

  \begin{frame}{Equivalencias e implicaciones entre cuantificadores}
    $\forall x\, P(x) \rightarrow \exists x\, P(x)$: Si para todo elmento del universo se cumple $P(x)$, tomando cualquiera
    se conluye lo requerido.

    $\exists! x\, P(x) \equiv \exists x\, P(x) \land \forall x\forall y ([P(x) \land P(y)] \rightarrow x = y)$, esto tiene 
    un sentido gramático: "La propiedad debe ser cumplida por almenos un elemento del universo y además, si hay dos elementos 
    que la cumplen, estos deben ser iguales".

    $\exists x\, \forall y\, P(x, y) \rightarrow \forall y\, \exists x \,P(x,y)$ 

    \begin{equation*}
      \begin{aligned}
        \exists x\, \forall y\, P(x, y) &\iff \forall y\, P(a, y), \text{ para algún } a \\
        
      \end{aligned}
    \end{equation*}
  \end{frame}

\end{document}
