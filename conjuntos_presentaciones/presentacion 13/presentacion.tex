\documentclass{beamer}
\newcommand{\pair}[2]{(#1, \, #2)}
\newcommand{\ineq}{%
  \mathrel{\ooalign{%
    $\in$\cr
    \hidewidth\raisebox{-0.6ex}{\rule{0.7em}{0.6pt}}\hidewidth\cr
  }}}

\begin{document}
  \begin{frame}
    \textit{Teorema (Recursión simple)}: Sea $A$ no vacío, $a \in A$ y $g: \mathbb{N} \times A \to A$,
    entonces existe una única recursión para $\pair{a}{g}$.
  \end{frame}

  \begin{frame}
    \textit{Teorema (Recursión parámétrica)}: Sean $A, P$ conjuntos no vacíos, $h: P \to A$
    y $g: P \times \mathbb{N} \times A \to A$, entonce esxiste una función tal que 
    
    \begin{itemize}
      \item $f(p, 0) = h(p)$.
    \end{itemize}
  \end{frame}

  \begin{frame}
    \textit{Teorema (Recursión parámétrica)}: Sean $A, P$ conjuntos no vacíos, $h: P \to A$
    y $g: P \times \mathbb{N} \times A \to A$, entonce esxiste una función tal que 
    
    \begin{itemize}
      \item $f(p, 0) = h(p)$.
      \item $f(p, n^+) = g(p, n, f(p, n))$.
    \end{itemize}
  \end{frame}

  \begin{frame}
    \textit{Definición}: Tomemos la función identidad $i_\mathbb{N}: \mathbb{N} \to \mathbb{N}$
    y 
    
    \begin{equation*}
      \begin{aligned}
        g: \mathbb{N} \times \mathbb{N} \times \mathbb{N} &\to \mathbb{N} \\
        g(p, n, r) &\to r^+ 
      \end{aligned}
    \end{equation*}
  \end{frame}

  \begin{frame}
    \textit{Definición}: Tomemos la función identidad $i_\mathbb{N}: \mathbb{N} \to \mathbb{N}$
    y 
    
    \begin{equation*}
      \begin{aligned}
        g: \mathbb{N} \times \mathbb{N} \times \mathbb{N} &\to \mathbb{N} \\
        g(p, n, r) &\to r^+ 
      \end{aligned}
    \end{equation*}

    Bajo estas condiciones, por el teorema de la recursión parámetrica vemos que existe la 
    función $+ : \mathbb{N} \times \mathbb{N} \to \mathbb{N}$, que por el teorema cumple que 
    para todo $p \in \mathbb{N}$:
  \end{frame}

  \begin{frame}
    \textit{Definición}: Tomemos la función identidad $i_\mathbb{N}: \mathbb{N} \to \mathbb{N}$
    y 
    
    \begin{equation*}
      \begin{aligned}
        g: \mathbb{N} \times \mathbb{N} \times \mathbb{N} &\to \mathbb{N} \\
        g(p, n, r) &\to r^+ 
      \end{aligned}
    \end{equation*}

    Bajo estas condiciones, por el teorema de la recursión parámetrica vemos que existe la 
    función $+ : \mathbb{N} \times \mathbb{N} \to \mathbb{N}$, que por el teorema cumple que 
    para todo $p \in \mathbb{N}$:

    \begin{itemize}
      \item $+(p, 0) = i_\mathbb{N}(p) = p$.
    \end{itemize}
  \end{frame}

  \begin{frame}
    \textit{Definición}: Tomemos la función identidad $i_\mathbb{N}: \mathbb{N} \to \mathbb{N}$
    y 
    
    \begin{equation*}
      \begin{aligned}
        g: \mathbb{N} \times \mathbb{N} \times \mathbb{N} &\to \mathbb{N} \\
        g(p, n, r) &\to r^+ 
      \end{aligned}
    \end{equation*}

    Bajo estas condiciones, por el teorema de la recursión parámetrica vemos que existe la 
    función $+ : \mathbb{N} \times \mathbb{N} \to \mathbb{N}$, que por el teorema cumple que 
    para todo $p \in \mathbb{N}$:

    \begin{itemize}
      \item $+(p, 0) = i_\mathbb{N}(p) = p$.
      \item $+(p, n^+) = g(p, n, +(p, n)) = (+(p, n))^+$
    \end{itemize}
  \end{frame}

  \begin{frame}
    \textit{Definición}: Tomemos la función identidad $i_\mathbb{N}: \mathbb{N} \to \mathbb{N}$
    y 
    
    \begin{equation*}
      \begin{aligned}
        g: \mathbb{N} \times \mathbb{N} \times \mathbb{N} &\to \mathbb{N} \\
        g(p, n, r) &\to r^+ 
      \end{aligned}
    \end{equation*}

    Bajo estas condiciones, por el teorema de la recursión parámetrica vemos que existe la 
    función $+ : \mathbb{N} \times \mathbb{N} \to \mathbb{N}$, que por el teorema cumple que 
    para todo $p \in \mathbb{N}$:

    \begin{itemize}
      \item $+(p, 0) = i_\mathbb{N}(p) = p$.
      \item $+(p, n^+) = g(p, n, +(p, n)) = (+(p, n))^+$
    \end{itemize}

    Esta función recibe el nombre de suma y para todo $p, n$ se denota como 
  \end{frame}

  \begin{frame}
    \textit{Definición}: Tomemos la función identidad $i_\mathbb{N}: \mathbb{N} \to \mathbb{N}$
    y 
    
    \begin{equation*}
      \begin{aligned}
        g: \mathbb{N} \times \mathbb{N} \times \mathbb{N} &\to \mathbb{N} \\
        g(p, n, r) &\to r^+ 
      \end{aligned}
    \end{equation*}

    Bajo estas condiciones, por el teorema de la recursión parámetrica vemos que existe la 
    función $+ : \mathbb{N} \times \mathbb{N} \to \mathbb{N}$, que por el teorema cumple que 
    para todo $p \in \mathbb{N}$:

    \begin{itemize}
      \item $+(p, 0) = i_\mathbb{N}(p) = p$.
      \item $+(p, n^+) = g(p, n, +(p, n)) = (+(p, n))^+$
    \end{itemize}

    Esta función recibe el nombre de suma y para todo $p, n$ se denota como 
    \[+(p, n) = p + n\]
  \end{frame}

  \begin{frame}
    \textit{Definición}: Tomemos la función $h: \mathbb{N} \to \mathbb{N}$, definida
    como $h(n) = 0$ para todo $n \in \mathbb{N}$.
  \end{frame}

  \begin{frame}
    \textit{Definición}: Tomemos la función $h: \mathbb{N} \to \mathbb{N}$, definida
    como $h(n) = 0$ para todo $n \in \mathbb{N}$. Tomemos la función 
    \begin{equation*}
      \begin{aligned}
        g: \mathbb{N} \times \mathbb{N} \times \mathbb{N} &\to \mathbb{N} \\
        g(p, n, r) &\to r + n 
      \end{aligned}
    \end{equation*}
  \end{frame}

  \begin{frame}
    \textit{Definición}: Tomemos la función $h: \mathbb{N} \to \mathbb{N}$, definida
    como $h(n) = 0$ para todo $n \in \mathbb{N}$. Tomemos la función 
    \begin{equation*}
      \begin{aligned}
        g: \mathbb{N} \times \mathbb{N} \times \mathbb{N} &\to \mathbb{N} \\
        g(p, n, r) &\to r + n 
      \end{aligned}
    \end{equation*}

    Por el teorema de la recursión parámetrica existe la función $\cdot: \mathbb{N} \times \mathbb{N} \to \mathbb{N}$
    que cumple las condiciones de teorema, es decir:
  \end{frame}

  \begin{frame}
    \textit{Definición}: Tomemos la función $h: \mathbb{N} \to \mathbb{N}$, definida
    como $h(n) = 0$ para todo $n \in \mathbb{N}$. Tomemos la función 
    \begin{equation*}
      \begin{aligned}
        g: \mathbb{N} \times \mathbb{N} \times \mathbb{N} &\to \mathbb{N} \\
        g(p, n, r) &\to r + p
      \end{aligned}
    \end{equation*}

    Por el teorema de la recursión parámetrica existe la función $\cdot: \mathbb{N} \times \mathbb{N} \to \mathbb{N}$
    que cumple las condiciones de teorema, es decir:

    \begin{itemize}
      \item $\cdot(p, 0) = h(0) = 0$.
    \end{itemize}
  \end{frame}

  \begin{frame}
    \textit{Definición}: Tomemos la función $h: \mathbb{N} \to \mathbb{N}$, definida
    como $h(n) = 0$ para todo $n \in \mathbb{N}$. Tomemos la función 
    \begin{equation*}
      \begin{aligned}
        g: \mathbb{N} \times \mathbb{N} \times \mathbb{N} &\to \mathbb{N} \\
        g(p, n, r) &\to r + p 
      \end{aligned}
    \end{equation*}

    Por el teorema de la recursión parámetrica existe la función $\cdot: \mathbb{N} \times \mathbb{N} \to \mathbb{N}$
    que cumple las condiciones de teorema, es decir:

    \begin{itemize}
      \item $\cdot(p, 0) = h(0) = 0$.
      \item $\cdot(p, n^+) = g(p, n, \cdot(p, n)) = \cdot(p, n) + p$.
    \end{itemize}
  \end{frame}

  \begin{frame}
    \textit{Teorema}: Para todo $n, m$ naturales, se cumple que 
    \[n \ineq m \iff (\exists k \in \mathbb{N})(n + k = m)\]
  \end{frame}

  \begin{frame}
    \textit{Definición}: Dado un conjunto $A$, definimos suc$(A)$ como 
    \[\text{suc}(A) := \bigcup_{n \in \mathbb{N}}\, ^n A\]
  \end{frame}

  \begin{frame}
    \textit{Teorema (Recursión completa)}: Sea $B$ no vacío, $g: \text{suc}(B) \to B$, entonces existe una 
    única función $f: \mathbb{N} \to B$ tal que:
    \[f(n) = g(f|_{n})\]
  \end{frame}

  \begin{frame}
  \textit{Teorema}: Sea $\pair{A}{\leq_A}$ un conjunto bien ordenado, sin máximo y en el que 
    todo subconjunto no vacío acotado superiormente tiene máximo, entonces $\pair{A}{\leq_A} \equiv \pair{\mathbb{N}}{\ineq}$
  \end{frame}
\end{document}
