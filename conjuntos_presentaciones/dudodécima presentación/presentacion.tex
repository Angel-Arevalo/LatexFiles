\documentclass{beamer}
\usepackage{amssymb}
\newcommand{\ineq}{%
  \mathrel{\ooalign{%
    $\in$\cr
    \hidewidth\raisebox{-0.6ex}{\rule{0.7em}{0.6pt}}\hidewidth\cr
  }}}
\newcommand{\pair}[2]{(#1,\, #2)}

\begin{document}
  \begin{frame}
    \textit{Teorema}: Sea $P(n)$ una propiedad tal que 
      \[[(\forall n \in m)(P(n))] \rightarrow P(m)\]

    entonces $(\forall m \in \mathbb{N})(P(n))$
  \end{frame}

  \begin{frame}
    \textit{Teorema}: Sea $P(n)$ una propiedad tal que 
      \[[(\forall n \in m)(P(n))] \rightarrow P(m)\]

    entonces $(\forall m \in \mathbb{N})(P(n))$
    \newline\newline

    Definiendo $H = \{n \in \mathbb{N}: \neg P(n)\}$ se concluye fácilmente.
  \end{frame}

  \begin{frame}
    \textit{Teorema}: Sea $P(n)$ una propiedad tal que
    \begin{itemize}
      \item $P(0)$
      \item $[(\forall n \ineq m)(P(n))] \rightarrow P(m^+)$
    \end{itemize}

    Entonces $P(n)$ es válida para todos los naturales.
  \end{frame}

  \begin{frame}
    \textit{Teorema}: Sea $\emptyset \neq B \subseteq \mathbb{N}$, entonces $\bigcap B \in \mathbb{N}$
  \end{frame}

  \begin{frame}
    \textit{Teorema}: Sea $\emptyset \neq B \subseteq \mathbb{N}$, entonces $\bigcap B \in \mathbb{N}$
    \newline\newline
    
    \textit{Teorema}: Sea $B \subseteq \mathbb{N}$, entonces $\bigcup B \in \mathbb{N}$ o $\bigcup B = \mathbb{N}$.
  \end{frame}

  \begin{frame}
    \textit{Teorema}: Sea $\emptyset \neq B \subseteq \mathbb{N}$, entonces $\bigcap B \in \mathbb{N}$
    \newline\newline
    
    \textit{Teorema}: Sea $B \subseteq \mathbb{N}$, entonces $\bigcup B \in \mathbb{N}$ o $\bigcup B = \mathbb{N}$.
    \newline\newline

    \textit{Teorema}: Todo conjunto no vacío de números naturales acotado superiormente, 
    tiene máximo.
  \end{frame}

  \begin{frame}
    \textit{Teorema}: Si $B \subset \mathbb{N}$ es acotado superiormente, $\bigcup B$ es 
    el máximo de $B$ y si es diferente de vacío, $\bigcap B$ es el mínimo de $B$ es el mínimo de $B$.
  \end{frame}

  \begin{frame}
    \textit{Teorema}: Si $B \subset \mathbb{N}$ es acotado superiormente, $\bigcup B$ es 
    el máximo de $B$ y si es diferente de vacío, $\bigcap B$ es el mínimo de $B$ es el mínimo de $B$.
    \newline\newline
    
    \textit{Teorema}: Si $0 \neq n \in \mathbb{N}$, entonces $\bigcap n = 0$ y $(\bigcup n)^+ = n$.
  \end{frame}

  \begin{frame}
    \textit{Definición}: Decimos que una función $f$ es una sucesión sobre $A$ si
    Dom$(f) = \mathbb{N}$ y Ran$(f) \subseteq A$. Se dice suseción finita 
    si Dom$(f) = n$, $n\in \mathbb{N}$ y Ran$(f) \subseteq A$.
  \end{frame}

  \begin{frame}
    \textit{Definición}: Decimos que una función $f$ es una sucesión sobre $A$ si
    Dom$(f) = \mathbb{N}$ y Ran$(f) \subseteq A$. Se dice suseción finita 
    si Dom$(f) = n$, $n\in \mathbb{N}$ y Ran$(f) \subseteq A$.


    Podemos adoptar la siguiente notación para una sucesión:
    \begin{equation*}
      \begin{aligned}
        f &= \{\pair{0}{f(0)}, \pair{1}{f(1)}, \cdots, \pair{n}{f(n)}, \cdots\} \\
          &= <f_0, f_1, \cdots, f_n, \cdots>
      \end{aligned}
    \end{equation*}
  \end{frame}

  \begin{frame}
    \textit{Definición}: Sea $A$ un conjunto no vacío, $a \in A$ y $g: \mathbb{N}\times A \to A$,
    decimos que una función $f: \mathbb{N} \to A$ es una recursión de $\pair{a}{g}$
    si:
    

    \begin{itemize}
      \item $f(0) = a$.
    \end{itemize}
  \end{frame}

  \begin{frame}
    \textit{Definición}: Sea $A$ un conjunto no vacío, $a \in A$ y $g: \mathbb{N}\times A \to A$,
    decimos que una función $f: \mathbb{N} \to A$ es una recursión de $\pair{a}{g}$
    si:
    

    \begin{itemize}
      \item $f(0) = a$.
      \item $f(n^+) = g(n, f(n))$.
    \end{itemize}
  \end{frame}

  \begin{frame}
    \textit{Definición}: Sea $A$ un conjunto no vacío, $a \in A$ y $g: \mathbb{N}\times A \to A$,
    decimos que una función $f: \mathbb{N} \to A$ es una recursión de $\pair{a}{g}$
    si:
    

    \begin{itemize}
      \item $f(0) = a$.
      \item $f(n^+) = g(n, f(n))$.
    \end{itemize}

    Se dice que $h: n\in\mathbb{N} \to A$ es una recursión finita de $\pair{a}{g}$ si 
  \end{frame}

  \begin{frame}
    \textit{Definición}: Sea $A$ un conjunto no vacío, $a \in A$ y $g: \mathbb{N}\times A \to A$,
    decimos que una función $f: \mathbb{N} \to A$ es una recursión de $\pair{a}{g}$
    si:
    

    \begin{itemize}
      \item $f(0) = a$.
      \item $f(n^+) = g(n, f(n))$.
    \end{itemize}

    Se dice que $h: n\in\mathbb{N} \to A$ es una recursión finita de $\pair{a}{g}$ si 

    \begin{itemize}
      \item $h(0) = a$.
    \end{itemize}
  \end{frame}

  \begin{frame}
    \textit{Definición}: Sea $A$ un conjunto no vacío, $a \in A$ y $g: \mathbb{N}\times A \to A$,
    decimos que una función $f: \mathbb{N} \to A$ es una recursión de $\pair{a}{g}$
    si:
    

    \begin{itemize}
      \item $f(0) = a$.
      \item $f(n^+) = g(n, f(n))$.
    \end{itemize}

    Se dice que $h: n\in\mathbb{N} \to A$ es una recursión finita de $\pair{a}{g}$ si 

    \begin{itemize}
      \item $h(0) = a$.
      \item $h(n) = g(n^-, f(n^-))$, siempre que $n^- \neq 0$.
    \end{itemize}
  \end{frame}

  \begin{frame}
    \textit{Teorema}: Dado $A \neq 0$, $a \in A$ y $g: \mathbb{N} \times A \to A$ y 
    $f$ una recursión sobre $\pair{a}{g}$, entonces $f|_n$ es 
    una recursión finita de $\pair{a}{g}$.
  \end{frame}

  \begin{frame}
    \textit{Teorema}: Dados $A \neq 0$, $a \in A$ y $g: \mathbb{N} \times A \to A$
    entonces para todo $0 \neq n \in \mathbb{N}$ existe 
    una única recursión finita de $\pair{a}{g}$ con dominio $n$.
  \end{frame}
\end{document}
