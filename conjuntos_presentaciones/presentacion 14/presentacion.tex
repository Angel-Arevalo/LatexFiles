\documentclass{beamer}
\newcommand{\pair}[2]{(#1, \, #2)}
\newcommand{\ineq}{%
  \mathrel{\ooalign{%
    $\in$\cr
    \hidewidth\raisebox{-0.6ex}{\rule{0.7em}{0.6pt}}\hidewidth\cr
  }}}

\begin{document}
  \begin{frame}
    \textit{Definición}: Un conjunto $A$ es ordinal si

    \begin{itemize}
      \item $A$ es transitivo.
    \end{itemize}
  \end{frame}

  \begin{frame}
    \textit{Definición}: Un conjunto $A$ es ordinal si

    \begin{itemize}
      \item $A$ es transitivo.
      \item $\pair{A}{\ineq}$ es un buen orden.
    \end{itemize}
  \end{frame}

  \begin{frame}
    \textit{Teorema}: Si $A$ es un ordinal, cumple las siguientes propiedades:
  \end{frame}

  \begin{frame}
    \textit{Teorema}: Si $A$ es un ordinal, cumple las siguientes propiedades:

    \begin{itemize}
      \item $A \not\in A$.
    \end{itemize}
  \end{frame}

  \begin{frame}
    \textit{Teorema}: Si $A$ es un ordinal, cumple las siguientes propiedades:

    \begin{itemize}
      \item $A \not\in A$.
      \item Si $x \in A$, entonces $x$ es ordinal.
    \end{itemize}
  \end{frame}

  \begin{frame}
    \textit{Teorema}: Si $A$ es un ordinal, cumple las siguientes propiedades:

    \begin{itemize}
      \item $A \not\in A$.
      \item Si $x \in A$, entonces $x$ es ordinal.
      \item Si $A^+ = \{A\}$, entonces $A = \emptyset$.
    \end{itemize}
  \end{frame}

  \begin{frame}
    \textit{Teorema}: Si $A$ es un ordinal, cumple las siguientes propiedades:

    \begin{itemize}
      \item $A \not\in A$.
      \item Si $x \in A$, entonces $x$ es ordinal.
      \item Si $A^+ = \{A\}$, entonces $A = \emptyset$.
      \item $A^+$ es ordinal.
    \end{itemize}
  \end{frame}

  \begin{frame}
    \textit{Teorema}: Todo natural es ordinal.
  \end{frame}

  \begin{frame}
    \textit{Teorema}: Todo natural es ordinal.
    \newline\newline

    \textit{Teorema}: $A$ es un número natural si y sólo si $A$ es un ordinal 
    en el que todo subconjunto no vacío tiene máximo y mínimo.
  \end{frame}

  \begin{frame}
    \textit{Definición}: Un conjunto $x$ es un número natural si  
  \end{frame}

  \begin{frame}
    \textit{Definición}: Un conjunto $x$ es un número natural si

    \begin{itemize}
      \item $x$ es transitivo.
    \end{itemize}
  \end{frame}

  \begin{frame}
    \textit{Definición}: Un conjunto $x$ es un número natural si

    \begin{itemize}
      \item $x$ es transitivo.
      \item $\pair{x}{\in}$ es un orden estricto y total  en $x$.
    \end{itemize}
  \end{frame}

  \begin{frame}
    \textit{Definición}: Un conjunto $x$ es un número natural si

    \begin{itemize}
      \item $x$ es transitivo.
      \item $\pair{x}{\in}$ es un orden estricto y total  en $x$.
      \item Todo subconjunto no vacío de $x$ tiene máximo y mínimo.
    \end{itemize}
  \end{frame}

  \begin{frame}
    \textit{Teorema}: $\emptyset$ es natural.
  \end{frame}

  \begin{frame}
    \textit{Teorema}: $\emptyset$ es natural.
    \newline\newline

    \textit{Teorema}: Si $a$ es natural, $a^+$ es natural.
  \end{frame}

  \begin{frame}
    \textit{Teorema}: $\emptyset$ es natural.
    \newline\newline

    \textit{Teorema}: Si $a$ es natural, $a^+$ es natural.
    \newline\newline

    \textit{Teorema}: Si $m \in x$ y $x$ es natural, entonces $m$ es natural.
  \end{frame}

  \begin{frame}
    \textit{Teorema}: $\emptyset$ es natural.
    \newline\newline

    \textit{Teorema}: Si $a$ es natural, $a^+$ es natural.
    \newline\newline

    \textit{Teorema}: Si $m \in x$ y $x$ es natural, entonces $m$ es natural.
    \newline\newline

    \textit{Teorema}: Si $x$ es natural, $x \not\in x$.
  \end{frame}

  \begin{frame}
    \textit{Teorema}: Si $x$ es natural y $A$ es un conjunto inductivo, 
    $x \in A$.
  \end{frame}

  \begin{frame}
    \textit{Teorema}: Si $x$ es natural y $A$ es un conjunto inductivo, 
    $x \in A$.
    \newline\newline

    Dado $A$ inductivo, definimos el conjunto de los números naturales como 
    \[\mathbb{N} := \{x \in A: x \text{ es natural}\}\]
  \end{frame}
\end{document}
