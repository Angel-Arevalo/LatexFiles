\documentclass{beamer}
\usepackage{amssymb}
\usepackage{amsmath}
\newcommand{\ineq}{%
  \mathrel{\ooalign{%
    $\in$\cr
    \hidewidth\raisebox{-0.6ex}{\rule{0.7em}{0.6pt}}\hidewidth\cr
  }}}

\begin{document}
    \begin{frame}{Axiomas de Peano}
      Dado un conjunto $N$ y  una función $S: N\to N$, decimos que
      $(N, i, S)$ es el conjunto de los números naturales 
      si se cumplen los siguientes axiomas:
      \begin{itemize}
        \item $\exists i \in N$.
      \end{itemize}
    \end{frame}

    \begin{frame}{Axiomas de Peano}
      Dado un conjunto $N$ y  una función $S: N\to N$, decimos que
      $(N, i, S)$ es el conjunto de los números naturales 
      si se cumplen los siguientes axiomas:
      \begin{itemize}
        \item $\exists i \in N$.
        \item Si $n \in N$ entonces $S(n) \in N$.
      \end{itemize}
    \end{frame}

    \begin{frame}{Axiomas de Peano}
      Dado un conjunto $N$ y  una función $S: N\to N$, decimos que
      $(N, i, S)$ es el conjunto de los números naturales 
      si se cumplen los siguientes axiomas:
      \begin{itemize}
        \item $\exists i \in N$.
        \item Si $n \in N$ entonces $S(n) \in N$.
        \item para todo $n \in N$, $S(n) \neq i$.
      \end{itemize}
    \end{frame}

    \begin{frame}{Axiomas de Peano}
      Dado un conjunto $N$ y  una función $S: N\to N$, decimos que
      $(N, i, S)$ es el conjunto de los números naturales 
      si se cumplen los siguientes axiomas:
      \begin{itemize}
        \item $\exists i \in N$.
        \item Si $n \in N$ entonces $S(n) \in N$.
        \item para todo $n \in N$, $S(n) \neq i$.
        \item Si $n, m \in N$ y $S(n) = S(m)$ entonces $n = m$.
      \end{itemize}
    \end{frame}

    \begin{frame}{Axiomas de Peano}
      Dado un conjunto $N$ y  una función $S: N\to N$, decimos que
      $(N, i, S)$ es el conjunto de los números naturales 
      si se cumplen los siguientes axiomas:
      \begin{itemize}
        \item $\exists i \in N$.
        \item Si $n \in N$ entonces $S(n) \in N$.
        \item para todo $n \in N$, $S(n) \neq i$.
        \item Si $n, m \in N$ y $S(n) = S(m)$ entonces $n = m$.
        \item Si $T \subseteq N$ tal que\newline
              1) $i \in T$
      \end{itemize}
    \end{frame}

    \begin{frame}{Axiomas de Peano}
      Dado un conjunto $N$ y  una función $S: N\to N$, decimos que
      $(N, i, S)$ es el conjunto de los números naturales 
      si se cumplen los siguientes axiomas:
      \begin{itemize}
        \item $\exists i \in N$.
        \item Si $n \in N$ entonces $S(n) \in N$.
        \item para todo $n \in N$, $S(n) \neq i$.
        \item Si $n, m \in N$ y $S(n) = S(m)$ entonces $n = m$.
        \item Si $T \subseteq N$ tal que\newline
              1) $i \in T$\newline
              2) Si $n \in T$ entonces $S(n) \in T$\newline

              Se concluye que $T = N$.
      \end{itemize}
    \end{frame}

    \begin{frame}{Axiomas de Peano}
      Dado un conjunto $N$ y  una función $S: N\to N$, decimos que
      $(N, i, S)$ es el conjunto de los números naturales 
      si se cumplen los siguientes axiomas:
      \begin{itemize}
        \item $\exists i \in N$.
        \item Si $n \in N$ entonces $S(n) \in N$.
        \item para todo $n \in N$, $S(n) \neq i$.
        \item Si $n, m \in N$ y $S(n) = S(m)$ entonces $n = m$.
        \item Si $T \subseteq N$ tal que\newline
              1) $i \in T$\newline
              2) Si $n \in T$ entonces $S(n) \in T$\newline

              Se concluye que $T = N$.
      \end{itemize}
    \end{frame}

    \begin{frame}
        \textit{Definición}: Dado un conjunto $A$ denotamos el sucesor como
        \[A^+ := A \cup \{A\}\]
    \end{frame}

    \begin{frame}
        \textit{Definición}: Dado un conjunto $A$ denotamos el sucesor como
        \[A^+ := A \cup \{A\}\]

        \textit{Teorema}: Para todo conjunto $A$, $A \subset A^+$ y $A \in A^+$.
    \end{frame}

    \begin{frame}
        \textit{Definición}: Dado un conjunto $A$ denotamos el sucesor como
        \[A^+ := A \cup \{A\}\]

        \textit{Teorema}: Para todo conjunto $A$, $A \subset A^+$ y $A \in A^+$.

        \textit{Teorema}: Para todo conjunto $A$, $A^+ \neq \emptyset$.
    \end{frame}

    \begin{frame}
        \textit{Teorema}: Sean $A$ y $B$ conjuntos. $B \subseteq A$ entonces
        $B \subseteq A^+$.
    \end{frame}

    \begin{frame}
        \textit{Teorema}: Sean $A$ y $B$ conjuntos. $B \subseteq A$ entonces
        $B \subseteq A^+$.
        \newline\newline\newline\newline\newline\newline\newline\newline
        \textit{Nota}: El resultado es válido también para pertenencia.
    \end{frame}

    \begin{frame}
        \textit{Definición}: Dado un conjunto $A$ decimos que es inductivo si
        \begin{equation*}
            \begin{aligned}
                &\text{1)} \emptyset \in A \\
                &\text{2) Si } x \in A \text{ entonces } x^+ \in A
            \end{aligned}
        \end{equation*}
    \end{frame}

    \begin{frame}
        \textit{Teorema}: Dado un conjunto $B$ en el que sus elementos
        son conjuntos inductivos, entonces $\bigcap B$ es inductivo.
    \end{frame}

    \begin{frame}
        \textit{Teorema}: Dado un conjunto $B$ en el que sus elementos
        son conjuntos inductivos, entonces $\bigcap B$ es inductivo.


        \textit{Teorema}: Sea $A$ un conjunto. Si el conjunto 
            \[I_A = \{x \in P(A): x \text{ es inductivo}\}\]

        es diferente de vacío, entonces $\bigcap I_A$ es el mínimo de 
        $(I_A, \subseteq)$
    \end{frame}

    \begin{frame}
        \textit{Definición}: Si $A$ es un conjunto cualquiera, llamamos
        $\bigcap I_A$ al mínimo conjunto inductivo de $A$.
    \end{frame}
    
    \begin{frame}
        \textit{Definición}: Si $A$ es un conjunto cualquiera, llamamos
        $\bigcap I_A$ al mínimo conjunto inductivo de $A$.
        \newline\newline\newline
        \textit{Teorema}: Si $A$ y $B$ son conjuntos inductivos tales que 
        $I_A \neq \emptyset \neq I_B$, entonces sus mínimos conjuntos inductivos
        son iguales.
    \end{frame}

    \begin{frame}
        Hasta el momento hemos definido los conjuntos inductivos pero no sabemos que 
        existe.
    \end{frame}

    \begin{frame}
        Hasta el momento hemos definido los conjuntos inductivos pero no sabemos que 
        existe.
        \newline\newline\newline

        \textit{Axioma 7}: Existe al menos un conjunto inductivo.
    \end{frame}

    \begin{frame}
        Hasta el momento hemos definido los conjuntos inductivos pero no sabemos que 
        existe.
        \newline\newline\newline

        \textit{Axioma 7}: Existe al menos un conjunto inductivo.
        \[\exists N\, (\emptyset \in N \land \forall x\,(x\in I \rightarrow x\cup \{x\} \in N))\]
    \end{frame}

    \begin{frame}
        Hasta el momento hemos definido los conjuntos inductivos pero no sabemos que 
        existe.
        \newline\newline\newline

        \textit{Axioma 7}: Existe al menos un conjunto inductivo.
        \[\exists N\, (\emptyset \in N \land \forall x\,(x\in N \rightarrow x\cup \{x\} \in N))\]


        \textit{Nota}: El conjunto asegurado por el axioma no es el conjunto de los
        números naturales.
    \end{frame}

    \begin{frame}
        \textit{Teorema (Números naturales)}: Existe un único conjunto 
        inductivo que es el mínimo de cualquier conjunto inductivo.
    \end{frame}

    \begin{frame}
        \textit{Teorema (Números naturales)}: Existe un único conjunto 
        inductivo que es el mínimo de cualquier conjunto inductivo.
        \newline\newline\newline

        \textit{Definición}: Al conjunto asegurado por el teorema anterior
        le denominamos $\mathbb{N}$.
    \end{frame}

    \begin{frame}
        \textit{Teorema (Números naturales)}: Existe un único conjunto 
        inductivo que es el mínimo de cualquier conjunto inductivo.
        \newline\newline\newline

        \textit{Definición}: Al conjunto asegurado por el teorema anterior
        le denominamos $\mathbb{N}$.


        \textit{Teorema}: Para todo conjunto inductivo $A$, $\mathbb{N} \subseteq A$. 
    \end{frame}

    \begin{frame}
        \textit{Definición}: Decimos que un conjunto $A$ es transitivo, si 
        \[(\forall a\in A)\, (a \in A \rightarrow a \subseteq A)\]
    \end{frame}

    \begin{frame}
        \textit{Definición}: Decimos que un conjunto $A$ es transitivo, si 
        \[(\forall a\in A)\, (a \in A \rightarrow a \subseteq A)\]
        \newline\newline\newline

        \textit{Teorema}: Si $A$ es transitivo, entonces $A^+$ es transitivo.
    \end{frame}

    \begin{frame}
        \textit{Definición}: Decimos que un conjunto $A$ es transitivo, si 
        \[(\forall a\in A)\, (a \in A \rightarrow a \subseteq A)\]
        \newline\newline\newline

        \textit{Teorema}: Si $A$ es transitivo, entonces $A^+$ es transitivo.
        \newline\newline\newline

        \textit{Teorema}: Si $A$ es transitivo, $\bigcup A^+ = A$
    \end{frame}

    \begin{frame}
      \textit{Teorema}: 
      \begin{itemize}
        \item Si $n \in \mathbb{N}$ entonces $n = 0$ o $0 \in n$.
      \end{itemize}
    \end{frame}

    \begin{frame}
      \textit{Teorema}: 
      \begin{itemize}
        \item Si $n \in \mathbb{N}$ entonces $n = 0$ o $0 \in n$.
        \item $\mathbb{N}$ es transitivo.
      \end{itemize}
    \end{frame}

    \begin{frame}
      \textit{Teorema}: 
      \begin{itemize}
        \item Si $n \in \mathbb{N}$ entonces $n = 0$ o $0 \in n$.
        \item $\mathbb{N}$ es transitivo.
        \item $n \in \mathbb{N}$ entonces $n$ es transitivo.
      \end{itemize}
    \end{frame}

    \begin{frame}
      \textit{Teorema}: 
      \begin{itemize}
        \item Si $n \in \mathbb{N}$ entonces $n = 0$ o $0 \in n$.
        \item $\mathbb{N}$ es transitivo.
        \item $n \in \mathbb{N}$ entonces $n$ es transitivo.
        \item $n \in \mathbb{N}$ y $n \neq 0$ entonces existe $k \in \mathbb{N}$ tal que $k^+ = n$.
      \end{itemize}
    \end{frame}

    \begin{frame}
      \textit{Teorema}: 
      \begin{itemize}
        \item Si $n \in \mathbb{N}$ entonces $n = 0$ o $0 \in n$.
        \item $\mathbb{N}$ es transitivo.
        \item $n \in \mathbb{N}$ entonces $n$ es transitivo.
        \item $n \in \mathbb{N}$ y $n \neq 0$ entonces existe $k \in \mathbb{N}$ tal que $k^+ = n$.
      \end{itemize}

      \textit{Teorema}: $\mathbb{N}$ cumple los axiomas de Peano.
    \end{frame}
    
    \begin{frame}
      \textit{Teorema} La pertenecia en $N$ es transitiva y a simétrica.
    \end{frame}

    \begin{frame}
      \textit{Teorema} La pertenecia en $N$ es transitiva y a simétrica.
      \begin{equation*}
        \begin{aligned}
          1) n \in m \text{ y } m \in j \rightarrow n \in j
        \end{aligned}
      \end{equation*}
    \end{frame}

    \begin{frame}
      \textit{Teorema} La pertenecia en $N$ es transitiva y a simétrica.
      \begin{equation*}
        \begin{aligned}
          1) n \in m \text{ y } m \in j \rightarrow n \in j \\
          2) n \in m \rightarrow m \not\in n
        \end{aligned}
      \end{equation*}
    \end{frame}

    \begin{frame}
      \textit{Teorema} La pertenecia en $N$ es transitiva y a simétrica.
      \begin{equation*}
        \begin{aligned}
          1) n \in m \text{ y } m \in j \rightarrow n \in j \\
          2) n \in m \rightarrow m \not\in n
        \end{aligned}
      \end{equation*}

      \textit{Nota}: $(N,\, \in)$ es un orden estricto.
    \end{frame}

    \begin{frame}
      \textit{Teorema}: Para todo $n, m$ naturales, si $n \in m$ entonces 
      $n^+ \in m$ o $n^+ = m$.
    \end{frame}

    \begin{frame}
      \textit{Definición}: Dados dos conjuntos $A$ y $B$, decimos que $A \ineq B$ si y sólo
      si $A \in B$ o $A = B$.
    \end{frame}

    \begin{frame}
      \textit{Definición}: Dados dos conjuntos $A$ y $B$, decimos que $A \ineq B$ si y sólo
      si $A \in B$ o $A = B$.


      \textit{Teorema}: $(\mathbb{N},\, \ineq)$ es un buen orden.
    \end{frame}
\end{document}
