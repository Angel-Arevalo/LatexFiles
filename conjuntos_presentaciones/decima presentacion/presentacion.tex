\documentclass{beamer}
\newcommand{\pair}[2]{(#1, \, #2)}

\begin{document}
  \begin{frame}
    \textit{Teorema}: Sea $\pair{A}{\leq}$ un orden y $a, b \in A$ tal que $a < b$, entonces
    \[S^<(a) \subseteq S^\leq(a) \subseteq S^<(b) \subseteq S^\leq(b)\]
  \end{frame}

  \begin{frame}
    \textit{Definicion}: Sea $\pair{A}{\leq_A}$ y $\pair{B}{\leq_B}$ ordenes. Decimos que son ordenes isomorfos si existe
    una función $f: A \to B$ biyectiva tal que para todo $x, y \in A$
      \[f(x) \leq_B f(y) \iff x \leq_A y\]

    Se denota  $\pair{A}{\leq_A} \equiv \pair{B}{\leq_B}$ 
  \end{frame}

  \begin{frame}
    \textit{Teorema}: Si $\pair{A}{\leq_A} \equiv \pair{B}{\leq_B}$ entonces $\pair{A}{\leq_A}$ es un orden total 
    si y sólo si $\pair{B}{\leq_B}$ es orden total.
  \end{frame}

  \begin{frame}
    \textit{Teorema}: Si $\pair{A}{\leq_A} \equiv \pair{B}{\leq_B}$ entonces $\pair{A}{\leq_A}$ es un orden total 
    si y sólo si $\pair{B}{\leq_B}$ es orden total.
    \newline\newline
    \textit{Teorema}: Si $\pair{A}{\leq_A} \equiv \pair{B}{\leq_B}$ entonces $\pair{A}{\leq_A}$ es buen orden
    si y sólo si $\pair{B}{\leq_B}$ es buen orden.
  \end{frame}

  \begin{frame}
    \textit{Teorema}: 
    \begin{itemize}
      \item $\pair{A}{\leq_A} \equiv \pair{A}{\leq_A}$.
    \end{itemize}
  \end{frame}

  \begin{frame}
    \textit{Teorema}: 
    \begin{itemize}
      \item $\pair{A}{\leq_A} \equiv \pair{A}{\leq_A}$.
      \item $\pair{A}{\leq_A} \equiv \pair{B}{\leq_B} \land \pair{B}{\leq_B} \equiv \pair{C}{\leq_C}$ entonces 
            $\pair{A}{\leq_A} \equiv \pair{C}{\leq_C}$
    \end{itemize}
  \end{frame}

  \begin{frame}
    \textit{Teorema}: 
    \begin{itemize}
      \item $\pair{A}{\leq_A} \equiv \pair{A}{\leq_A}$.
      \item $\pair{A}{\leq_A} \equiv \pair{B}{\leq_B} \land \pair{B}{\leq_B} \equiv \pair{C}{\leq_C}$ entonces 
            $\pair{A}{\leq_A} \equiv \pair{C}{\leq_C}$

      \item $\pair{A}{\leq_A} \equiv \pair{B}{\leq_B}$ entonces $\pair{B}{\leq_B} \equiv \pair{A}{\leq_A}$
    \end{itemize}
  \end{frame}

  \begin{frame}
    \textit{Teorema}: Sea $A$ un conjunto y $\widehat{A}$ y $\bigstar$ los conjuntos definidos en la novena 
    presentacion. La relación de orden isomorfismo es una relación de equivalencia sobre $\widehat{A}$ y $\bigstar$.
  \end{frame}

  \begin{frame}
    \textit{Teorema}: Si $\pair{A}{\leq_A} \equiv \pair{B}{\leq_B}$ y $\pair{C}{\leq_C} \equiv \pair{D}{\leq_D}$
    entonces 
      \[\pair{A\times C}{\leq_1} \equiv \pair{B \times D}{\leq_2}\]
    donde $\leq_1$ es el orden lexicográfico sobre $A \times C$ y $\leq_2$ es el orden lexicográfico 
    sobre $B \times D$.
  \end{frame}

  \begin{frame}
    \textit{Teorema}: Si $\pair{A}{\leq_A} \equiv \pair{B}{\leq_B}$ entonces 
    \begin{itemize}
      \item $\pair{S^{<_A}(a)}{\leq_A} \equiv \pair{S^{<_B}(f(a))}{\leq_B}$
    \end{itemize}
  \end{frame}

  \begin{frame}
    \textit{Teorema}: Si $\pair{A}{\leq_A} \equiv \pair{B}{\leq_B}$ entonces 
    \begin{itemize}
      \item $\pair{S^{<_A}(a)}{\leq_A} \equiv \pair{S^{<_B}(f(a))}{\leq_B}$
      \item $\pair{S^{\leq_A}(a)}{\leq_A} \equiv \pair{S^{\leq_B}(f(a))}{\leq_B}$
    \end{itemize}
  \end{frame}

  \begin{frame}
    \textit{Teorema}: Sea $f: A \to B$, $\pair{A}{\leq_A}$ y $\pair{B}{\leq_B}$ conjuntos ordenados,
    decimos que 

    \begin{itemize}
      \item $f$ es creciente si $(\forall x,y \in A)(x \leq_A y \rightarrow f(x) \leq_B f(y)$.
    \end{itemize}
  \end{frame}


  \begin{frame}
    \textit{Teorema}: Sea $f: A \to B$, $\pair{A}{\leq_A}$ y $\pair{B}{\leq_B}$ conjuntos ordenados,
    decimos que 

    \begin{itemize}
      \item $f$ es creciente si $(\forall x,y \in A)(x \leq_A y \rightarrow f(x) \leq_B f(y)$.
      \item $f$ es decreciente si $(\forall x,y \in A)(x \leq_A y \rightarrow f(y) \leq_B f(x)$.
    \end{itemize}
  \end{frame}


  \begin{frame}
    \textit{Teorema}: Sea $f: A \to B$, $\pair{A}{\leq_A}$ y $\pair{B}{\leq_B}$ conjuntos ordenados,
    decimos que 

    \begin{itemize}
      \item $f$ es creciente si $(\forall x,y \in A)(x \leq_A y \rightarrow f(x) \leq_B f(y)$.
      \item $f$ es decreciente si $(\forall x,y \in A)(x \leq_A y \rightarrow f(y) \leq_B f(x)$.
      \item $f$ es estrictamente creciente si $(\forall x,y \in A)(x <_A y \rightarrow f(y) <_B f(x)$.
    \end{itemize}
  \end{frame}


  \begin{frame}
    \textit{Teorema}: Sea $f: A \to B$, $\pair{A}{\leq_A}$ y $\pair{B}{\leq_B}$ conjuntos ordenados,
    decimos que 

    \begin{itemize}
      \item $f$ es creciente si $(\forall x,y \in A)(x \leq_A y \rightarrow f(x) \leq_B f(y)$.
      \item $f$ es decreciente si $(\forall x,y \in A)(x \leq_A y \rightarrow f(y) \leq_B f(x)$.
      \item $f$ es estrictamente creciente si $(\forall x,y \in A)(x <_A y \rightarrow f(y) <_B f(x)$.
      \item $f$ es estrictamente decreciente si $(\forall x,y \in A)(x <_A y \rightarrow f(y) >_B f(x)$.
    \end{itemize}
  \end{frame}

  \begin{frame}
    \textit{Teorema}: Si $f: A \to B$ es creciente e inyectiva, entonces es estrictamente creciente.
  \end{frame}

  \begin{frame}
    \textit{Teorema}: Si $\pair{A}{\leq_A}$ y $\pair{B}{\leq}$ son conjuntos ordenados, $\leq_A$ es total y 
    $f$ estrictamente creciente, entonces $f$ es inyectiva.
  \end{frame}

  \begin{frame}
    \textit{Teorema}: Si $\pair{A}{\leq_A}$ y $\pair{B}{\leq_B}$ son ordenes, $\leq_A$ es total y $f: A \to B$
    es inyectiva y creciente, entonces $f$ es isomorfismo sobre su imagen.
  \end{frame}

  \begin{frame}
    \textit{Teorema (del diablo)}: Sean $\pair{A}{\leq_A}$ y $\pair{B}{\leq_B}$ conjuntos ordenados tales que 
    \begin{equation*}
      \begin{aligned}
        (\forall x, w \in A)(x \neq w \rightarrow \pair{S^{<_A}(x)}{\leq_A} \not\equiv \pair{S^{<_A}(w)}{\leq_A}) \\
        (\forall z, y \in A)(z \neq y \rightarrow \pair{S^{<_B}(z)}{\leq_B} \not\equiv \pair{S^{<_B}(y)}{\leq_B})
      \end{aligned}
    \end{equation*}

    entonces se cumple una de las siguientes afirmaciones 
    \begin{itemize}
      \item $\pair{A}{\leq_A} \equiv \pair{B}{\leq_B}$.
      \item $\pair{A}{\leq_A} \equiv \pair{S^{<_B}(b)}{\leq_B}$ para algún $b \in B$.
      \item $\pair{S^{<_A}(a)}{\leq_A} \equiv \pair{B}{\leq_B}$ para algún $a \in A$.
    \end{itemize}
  \end{frame}
\end{document}
