\documentclass{beamer}
\newcommand{\pair}[2]{(#1,\, #2)}


\begin{document}
  
  \begin{frame}{Ordenes}
    \textit{Definición}: Sea $\preceq$ una relación en un conjunto $A$. Decimos que 
    $\preceq$ es una relación de orden si es reflexiva, antisimétrica y transitiva.
  \end{frame}

  \begin{frame}{Ordenes}
    \textit{Definición}: Sea $\preceq$ una relación en un conjunto $A$. Decimos que 
    $\preceq$ es una relación de orden si es reflexiva, antisimétrica y transitiva.
    \newline\newline

    \textit{Teorema}: Si $\preceq$ es una relación de orden en $A$, $\preceq^{-1}$ es una relación de orden en $A$. 
  \end{frame}

  \begin{frame}{Orden estricto}
    \textit{Definición}: Decimos que una relación $\prec$ es un orden estricto en $A$ si es 
    una relación asimétrica y transitiva.
  \end{frame}

  \begin{frame}
    \textit{Definición}: Dado un conjunto $A$, denominamos el par $(A, \preceq)$ conjunto ordenado
    siempre que $\preceq$ sea un relación en $A$ y una relación de orden.
  \end{frame}

  \begin{frame}
    \textit{Teorema}: Sea $(A, \preceq)$ un conjunto ordenado. Para todo $B \subseteq A$, 
    $(B, \preceq_1)$ es un conjunto ordenado, donde $\preceq_1 = \{(x, y) \in \preceq: x \in B \land y \in B\}$.
  \end{frame}

  \begin{frame}
    \textit{Teorema}: Sea $(A, \preceq)$ un conjunto ordenado. Para todo $B \subseteq A$, 
    $(B, \preceq_1)$ es un conjunto ordenado, donde $\preceq_1 = \{(x, y) \in \preceq: x \in B \land y \in B\}$.
    \newline\newline

    \textit{antisimetría}: Si $\pair{x}{y}, \pair{y}{x} \in \preceq_1$, es porque 
    $x, y \in B$ y $\pair{x}{y}, \pair{y}{x} \in \preceq$, es decir, $x = y$ y $x \in B$.
  \end{frame}

  \begin{frame}
    \textit{Teorema}: Sea $(A, \preceq)$ un conjunto ordenado. Para todo $B \subseteq A$, 
    $(B, \preceq_1)$ es un conjunto ordenado, donde $\preceq_1 = \{(x, y) \in \preceq: x \in B \land y \in B\}$.
    \newline\newline

    \textit{antisimetría}: Si $\pair{x}{y}, \pair{y}{x} \in \preceq_1$, es porque 
    $x, y \in B$ y $\pair{x}{y}, \pair{y}{x} \in \preceq$, es decir, $x = y$ y $x \in B$.
    \newline\newline

    \textit{Transitividad}: Si $\pair{x}{y}, \pair{y}{z} \in \preceq_1$, de nuevo, $x, y, z \in B$ y 
    $\pair{x}{y}, \pair{y}{z} \in \preceq$, luego $x, z \in B$ y $\pair{x}{z} \in B$.
  \end{frame}

  \begin{frame}
    \textit{Teorema}: Sea $(A, \preceq)$ un conjunto ordenado. Para todo $B \subseteq A$, 
    $(B, \preceq_1)$ es un conjunto ordenado, donde $\preceq_1 = \{(x, y) \in \preceq: x \in B \land y \in B\}$.
    \newline\newline

    \textit{antisimetría}: Si $\pair{x}{y}, \pair{y}{x} \in \preceq_1$, es porque 
    $x, y \in B$ y $\pair{x}{y}, \pair{y}{x} \in \preceq$, es decir, $x = y$ y $x \in B$.
    \newline\newline

    \textit{Transitividad}: Si $\pair{x}{y}, \pair{y}{z} \in \preceq_1$, de nuevo, $x, y, z \in B$ y 
    $\pair{x}{y}, \pair{y}{z} \in \preceq$, luego $x, z \in B$ y $\pair{x}{z} \in B$.
    \newline\newline

    \textit{Reflexividad}: Inmediata.
  \end{frame}

  \begin{frame}
    \textit{Teorema}: Sean $\pair{A}{\preceq_1}$ y $\pair{B}{\preceq_2}$ conjuntos ordenados, entonces 
    $\preceq := \{\pair{\pair{a}{b}}{\pair{c}{d}} \in (A \times B)\times (A \times B): (a \prec_1 c) \lor (a = c \land b \preceq_2 d)\}$
    es un orden sobre $A\times B$.
  \end{frame}

  \begin{frame}
    \textit{Definición}: Sea $\pair{A}{\preceq}$ un orden, $B \subseteq A$, $z \in A$ y $x \in B$

    \begin{itemize}
      \item $x$ es el mínimo de $B$ si $(\forall y \in B)(x \preceq y)$.
    \end{itemize}
  \end{frame}

  \begin{frame}
    \textit{Definición}: Sea $\pair{A}{\preceq}$ un orden, $B \subseteq A$, $z \in A$ y $x \in B$

    \begin{itemize}
      \item $x$ es el mínimo de $B$ si $(\forall y \in B)(x \preceq y)$.
      \item $x$ es el máximo de $B$ si $(\forall y \in B)(y \preceq x)$.
    \end{itemize}
  \end{frame}

  \begin{frame}
    \textit{Definición}: Sea $\pair{A}{\preceq}$ un orden, $B \subseteq A$, $z \in A$ y $x \in B$

    \begin{itemize}
      \item $x$ es el mínimo de $B$ si $(\forall y \in B)(x \preceq y)$.
      \item $x$ es el máximo de $B$ si $(\forall y \in B)(y \preceq x)$.
      \item $x$ es mínimal de $B$ si $(\forall y \in B)(\neg (y \prec x))$.
    \end{itemize}
  \end{frame}

  \begin{frame}
    \textit{Definición}: Sea $\pair{A}{\preceq}$ un orden, $B \subseteq A$, $z \in A$ y $x \in B$

    \begin{itemize}
      \item $x$ es el mínimo de $B$ si $(\forall y \in B)(x \preceq y)$.
      \item $x$ es el máximo de $B$ si $(\forall y \in B)(y \preceq x)$.
      \item $x$ es mínimal de $B$ si $(\forall y \in B)(\neg (y \prec x))$.
      \item $x$ es máximal de $B$ si $(\forall y \in B)(\neg (x \prec y))$.
    \end{itemize}
  \end{frame}

  \begin{frame}
    \textit{Definición}: Sea $\pair{A}{\preceq}$ un orden, $B \subseteq A$, $z \in A$ y $x \in B$

    \begin{itemize}
      \item $x$ es el mínimo de $B$ si $(\forall y \in B)(x \preceq y)$.
      \item $x$ es el máximo de $B$ si $(\forall y \in B)(y \preceq x)$.
      \item $x$ es mínimal de $B$ si $(\forall y \in B)(\neg (y \prec x))$.
      \item $x$ es máximal de $B$ si $(\forall y \in B)(\neg (x \prec y))$.
      \item $z$ es cota inferior de $B$ si $(\forall y \in B)(z \preceq y)$.        
    \end{itemize}
  \end{frame}

  \begin{frame}
    \textit{Definición}: Sea $\pair{A}{\preceq}$ un orden, $B \subseteq A$, $z \in A$ y $x \in B$

    \begin{itemize}
      \item $x$ es el mínimo de $B$ si $(\forall y \in B)(x \preceq y)$.
      \item $x$ es el máximo de $B$ si $(\forall y \in B)(y \preceq x)$.
      \item $x$ es mínimal de $B$ si $(\forall y \in B)(\neg (y \prec x))$.
      \item $x$ es máximal de $B$ si $(\forall y \in B)(\neg (x \prec y))$.
      \item $z$ es cota inferior de $B$ si $(\forall y \in B)(z \preceq y)$.        
      \item $z$ es cota superior de $B$ si $(\forall y \in B)(y \preceq z)$.
    \end{itemize}
  \end{frame}

  \begin{frame}
    \textit{Definición}: Sea $\pair{A}{\preceq}$ un orden, $B \subseteq A$, $z \in A$ y $x \in B$

    \begin{itemize}
      \item $x$ es el mínimo de $B$ si $(\forall y \in B)(x \preceq y)$.
      \item $x$ es el máximo de $B$ si $(\forall y \in B)(y \preceq x)$.
      \item $x$ es mínimal de $B$ si $(\forall y \in B)(\neg (y \prec x))$.
      \item $x$ es máximal de $B$ si $(\forall y \in B)(\neg (x \prec y))$.
      \item $z$ es cota inferior de $B$ si $(\forall y \in B)(z \preceq y)$.        
      \item $z$ es cota superior de $B$ si $(\forall y \in B)(y \preceq z)$.
      \item $z$ es el supremo de $B$ si la menor de las cotas superiores.
    \end{itemize}
  \end{frame}

  \begin{frame}
    \textit{Definición}: Sea $\pair{A}{\preceq}$ un orden, $B \subseteq A$, $z \in A$ y $x \in B$

    \begin{itemize}
      \item $x$ es el mínimo de $B$ si $(\forall y \in B)(x \preceq y)$.
      \item $x$ es el máximo de $B$ si $(\forall y \in B)(y \preceq x)$.
      \item $x$ es mínimal de $B$ si $(\forall y \in B)(\neg (y \prec x))$.
      \item $x$ es máximal de $B$ si $(\forall y \in B)(\neg (x \prec y))$.
      \item $z$ es cota inferior de $B$ si $(\forall y \in B)(z \preceq y)$.        
      \item $z$ es cota superior de $B$ si $(\forall y \in B)(y \preceq z)$.
      \item $z$ es el supremo de $B$ si la menor de las cotas superiores.
      \item $z$ es el ínfimo de $B$ si es la mayor de las cotas superiores.
    \end{itemize}
  \end{frame}

  \begin{frame}
    Sea $\pair{A}{\preceq}$ un conjunto ordenado y $B \subseteq A$.
  \end{frame}

  \begin{frame}
    Sea $\pair{A}{\preceq}$ un conjunto ordenado y $B \subseteq A$.
    \newline\newline
    \textit{Teorema}: Si $B$ tiene mínimo, este es único.
  \end{frame}

  \begin{frame}
    Sea $\pair{A}{\preceq}$ un conjunto ordenado y $B \subseteq A$.
    \newline\newline
    \textit{Teorema}: Si $B$ tiene mínimo, este es único.
    \newline\newline
    \textit{Teorema}: Si $B$ tiene máximo, este es único.
  \end{frame}

  \begin{frame}
    Sea $\pair{A}{\preceq}$ un conjunto ordenado y $B \subseteq A$.
    \newline\newline
    \textit{Teorema}: Si $B$ tiene mínimo, este es único.
    \newline\newline
    \textit{Teorema}: Si $B$ tiene máximo, este es único.
    \newline\newline
    \textit{Teorema}: Si $B$ tiene supremo, este es único.
  \end{frame}

  \begin{frame}
    Sea $\pair{A}{\preceq}$ un conjunto ordenado y $B \subseteq A$.
    \newline\newline
    \textit{Teorema}: Si $B$ tiene mínimo, este es único.
    \newline\newline
    \textit{Teorema}: Si $B$ tiene máximo, este es único.
    \newline\newline
    \textit{Teorema}: Si $B$ tiene supremo, este es único.
    \newline\newline
    \textit{Teorema}: Si $B$ tiene ínfimo, este es único.
  \end{frame}

  \begin{frame}
    Sea $\pair{A}{\preceq}$ un conjunto ordenado y $B \subseteq A$.
    \newline\newline
    \textit{Teorema}: Si $B$ tiene mínimo, este es único.
    \newline\newline
    \textit{Teorema}: Si $B$ tiene máximo, este es único.
    \newline\newline
    \textit{Teorema}: Si $B$ tiene supremo, este es único.
    \newline\newline
    \textit{Teorema}: Si $B$ tiene ínfimo, este es único.
    \newline\newline

    \textit{Notación}: Se denotan $\text{min}(B)$, máx$(B)$, sup$(B)$ e inf$(B)$, respectivamente.
  \end{frame}

  \begin{frame}
    \textit{Definición}: Sea $(A, \preceq)$ un orden. Decimos que el orden es completo superiormente si 
    todo subconjunto de $A$ acotado superiormente tiene supremo. Decimos que es completo inferiormente 
    si $\pair{A}{\preceq^{-1}}$ es completo superiormente.
  \end{frame}

  \begin{frame}
    \textit{Definición}: Sea $(A, \preceq)$ un orden. Decimos que el orden es completo superiormente si 
    todo subconjunto de $A$ acotado superiormente tiene supremo. Decimos que es completo inferiormente 
    si $\pair{A}{\preceq^{-1}}$ es completo superiormente.
    \newline\newline

    \textit{Teorema}: $\pair{A}{\preceq}$ es completo inferiormente si y sólo si todo subconjunto 
    acotado inferiormente de $A$ tiene ínfimo.
  \end{frame}

  \begin{frame}
    \textit{Definición}: Sea $\pair{A}{\preceq}$ un conjunto ordenado, decimos que $B$ es una cadena
    en $A$ si
    \[(\forall x, y \in B)(x \preceq y \lor y \preceq x)\]
  \end{frame}

  \begin{frame}
    \textit{Teorema}: Sea $\pair{A}{\preceq}$ un orden. Si existe una cadena $\subseteq$-máximal y toda
    cadena de $\pair{A}{\preceq}$ está acotada superiormente, entonces existe un 
    elemento máximal en $\pair{A}{\preceq}$.
  \end{frame}

  \begin{frame}
    \textit{Teorema}: Sea $\pair{A}{\preceq}$ un orden. Si existe una cadena $\subseteq$-máximal y toda
    cadena de $\pair{A}{\preceq}$ está acotada superiormente, entonces existe un 
    elemento máximal en $\pair{A}{\preceq}$.
    \newline\newline

    Sea $B$ la cadena $\subseteq$-máximal. Probaremos que B tiene máximo. Supongamos que contrario
    $B$ no tenga máximo.
  \end{frame}

  \begin{frame}
    \textit{Teorema}: Sea $\pair{A}{\preceq}$ un orden. Si existe una cadena $\subseteq$-máximal y toda
    cadena de $\pair{A}{\preceq}$ está acotada superiormente, entonces existe un 
    elemento máximal en $\pair{A}{\preceq}$.
    \newline\newline

    Sea $B$ la cadena $\subseteq$-máximal. Probaremos que B tiene máximo. Supongamos que contrario
    $B$ no tenga máximo.
    \newline\newline

    Como $B$ es una cadena, entonces es acotada superiormente, luego existe un elemento
    $z \in A$ tal que para todo $y \in B$, $y \preceq z$. Como se asume que no tiene máximo
    entonces para cada $y \in B$ existe $x \in B$ tal que $y \prec x$.
  \end{frame}

  \begin{frame}
    \textit{Teorema}: Sea $\pair{A}{\preceq}$ un orden. Si existe una cadena $\subseteq$-máximal y toda
    cadena de $\pair{A}{\preceq}$ está acotada superiormente, entonces existe un 
    elemento máximal en $\pair{A}{\preceq}$.
    \newline\newline

    Sea $B$ la cadena $\subseteq$-máximal. Probaremos que B tiene máximo. Supongamos que contrario
    $B$ no tenga máximo.
    \newline\newline

    Como $B$ es una cadena, entonces es acotada superiormente, luego existe un elemento
    $z \in A$ tal que para todo $y \in B$, $y \preceq z$, de esto, podemos ver que 
    $z \notin B$ y como $B \subseteq B \cup \{z\} := C$, $C$ es una cadena, pero esto contradice
    que $B$ sea $\subseteq$-máximal.
  \end{frame}

  \begin{frame}
    \textit{Teorema}: Sea $\pair{A}{\preceq}$ un orden. Si existe una cadena $\subseteq$-máximal y toda
    cadena de $\pair{A}{\preceq}$ está acotada superiormente, entonces existe un 
    elemento máximal en $\pair{A}{\preceq}$.
    \newline\newline

    Sea $B$ la cadena $\subseteq$-máximal. Probaremos que B tiene máximo. Supongamos que contrario
    $B$ no tenga máximo.
    \newline\newline

    Como $B$ es una cadena, entonces es acotada superiormente, luego existe un elemento
    $z \in A$ tal que para todo $y \in B$, $y \preceq z$, de esto, podemos ver que 
    $z \notin B$ y como $B \subseteq B \cup \{z\} := C$, $C$ es una cadena, pero esto contradice
    que $B$ sea $\subseteq$-máximal.
    \newline\newline

    Tomando $x = \max(B)$, veamos que es elemento máximal.
  \end{frame}
\end{document}
