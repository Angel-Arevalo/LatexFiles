\documentclass{beamer}


\begin{document}
  \begin{frame}{Intersección  arbitraría de conjuntos}
    Como vimos en lo anterior, con el axioma 3 podemos garantizar la existencia de las operaciones
    conjuntistas $A\cap B$ y $A\setminus B$, para todo par de conjuntos $A$ y $B$, ahora garatizamos la 
    existencia de la intersección de una familia arbitraría de conjuntos.

    \textbf{Teorema 4:} Sea $A\neq \emptyset$, entonces existe un único conjunto $C$ tal que 
    $x\in C \iff \forall B\, (B\in A \rightarrow x\in B)$.
  \end{frame}

  \begin{frame}{Intersección  arbitraría de conjuntos}
    \textbf{Teorema 4:} Sea $A\neq \emptyset$, entonces existe un único conjunto $C$ tal que 
    $x\in C \iff \forall B\, (B\in A \rightarrow x\in B)$.

    Como $A$ es no vacío, entonces existe $D\in A$, definimos el conjunto $C = \{x\in D: \forall B\, (B\in A \rightarrow x\in B)\}$,
    garatizado por el axioma 3. Ahora probemos que este conjunto cumple que 
    $x\in C \iff \forall B\, (B\in A \rightarrow x\in B)$:
  \end{frame}

  \begin{frame}{Intersección  arbitraría de conjuntos}
        \textbf{Teorema 4:} Sea $A\neq \emptyset$, entonces existe un único conjunto $C$ tal que 
    $x\in C \iff \forall B\, (B\in A \rightarrow x\in B)$.

    Como $A$ es no vacío, entonces existe $D\in A$, definimos el conjunto $C = \{x\in D: \forall B\, (B\in A \rightarrow x\in B)\}$,
    garatizado por el axioma 3. Ahora probemos que este conjunto cumple que 
    $x\in C \iff \forall B\, (B\in A \rightarrow x\in B)$:

    \begin{itemize}
      \item $\Longrightarrow)$ 

        Si $x \in C$, cumple la propiedad con la que se le definió el conjunto arriba, luego
        $\forall B\, (B\in A \rightarrow x\in B)$
    \end{itemize}
  \end{frame}

  \begin{frame}{Intersección  arbitraría de conjuntos}
    \textbf{Teorema 4:} Sea $A\neq \emptyset$, entonces existe un único conjunto $C$ tal que 
    $x\in C \iff \forall B\, (B\in A \rightarrow x\in B)$.

    Como $A$ es no vacío, entonces existe $D\in A$, definimos el conjunto $C = \{x\in D: \forall B\, (B\in A \rightarrow x\in B)\}$,
    garatizado por el axioma 3. Ahora probemos que este conjunto cumple que 
    $x\in C \iff \forall B\, (B\in A \rightarrow x\in B)$:

    \begin{itemize}
      \item $\Longrightarrow)$ 

        Si $x \in C$, cumple la propiedad con la que se le definió el conjunto arriba, luego
        $\forall B\, (B\in A \rightarrow x\in B)$
      \item $\Longleftarrow)$

        Como $\forall B\, (B\in A \rightarrow x\in B)$, y como $D\in A$, entonces $x\in D$, luego,
        $x\in D \land \forall B\, (B\in A \rightarrow x\in B)$, por lo tanto $x\in C$
    \end{itemize}
  \end{frame}

  \begin{frame}{Intersección  arbitraría de conjuntos}
    \textbf{Teorema 4:} Sea $A\neq \emptyset$, entonces existe un único conjunto $C$ tal que 
    $x\in C \iff \forall B\, (B\in A \rightarrow x\in B)$.

    Ya probamos la existencia del conjunto $C$, pero en el argumento hecho no se garaiza la unicidad, 
    porque el conjunto $C$ podría ser distinto al tomar un $D$ diferente.
  \end{frame}

  \begin{frame}{Intersección arbitraría de conjuntos}
    \textbf{Teorema 4:} Sea $A\neq \emptyset$, entonces existe un único conjunto $C$ tal que 
    $x\in C \iff \forall B\, (B\in A \rightarrow x\in B)$.

    Si existen dos conjuntos, $C_1$ y $C_2$ que cumplen dichas condiciones, entonces 

    \begin{equation*}
      \begin{aligned}
        x \in C_1 &\iff \forall B\, (B\in A \rightarrow x\in B) \\
        &\iff x\in C_2
      \end{aligned}
    \end{equation*}

    Con esto queda culminada la prueba del teorema 4. Como se garatiza existencia y unicidad, podemos denotar con
    un nombre propio al conjunto $C$, que es $\bigcap A$, para $A\neq \emptyset$ y se dice que $x\in \bigcap A$ si y solamente si 
    $x\in D$, para todo $D\in A$.
  \end{frame}

  \begin{frame}{Algo de notación}
    En la demostración anterior se usó bastante la notación $\forall B\, (B\in A \rightarrow x\in B)$, esto es bastante 
    complejo tanto para leerlo como para escribirlo, en ese sentido, se hace la simplificación 
    $(\forall B\in A)\, (x\in B)$ que se lee exactamente igual.

    Asi mismo, para la notación $\exists B\, (\varphi_1(B, A) \land \varphi_2(x, B))$, se denota como 
    $\exists \varphi_1(B, A) (\varphi_2(x, B))$.
  \end{frame}

  \begin{frame}{Ejercicio}
    \textbf{Ejercicio:} Sea $A\neq \emptyset$. $\forall C$, $C\subseteq \bigcap A \Leftrightarrow (\forall B\in A)\, (C\subseteq B)$

    \begin{itemize}
      \item $\Longrightarrow)$

            Como $C \subseteq \bigcap A$, entonces, si $x\in C$, luego $x\in \bigcap A$, por
            la nota hecha anteriormente, sabemos que $x\in D$, para todo $D\in A$, es decir,
            $C \subseteq D$, para todo $D\in A$.
    \end{itemize}
  \end{frame}

  \begin{frame}{Ejercicio}
    \textbf{Ejercicio:} Sea $A\neq \emptyset$. $\forall C$, $C\subseteq \bigcap A \Leftrightarrow (\forall B\in A)\, (C\subseteq B)$

    \begin{itemize}
      \item $\Longrightarrow)$

            Como $C \subseteq \bigcap A$, entonces, si $x\in C$, luego $x\in \bigcap A$, por
            la nota hecha anteriormente, sabemos que $x\in D$, para todo $D\in A$, es decir,
            $C \subseteq D$, para todo $D\in A$.

      \item $\Longleftarrow)$

            Ahora, si $x\in C$, sabemos que para todo $B\in A$, $x\in B$, entonces $x\in \bigcap A$.
    \end{itemize}
  \end{frame}

  \begin{frame}{Axioma del par}
    \[\forall u\, \forall v\, \exists A \,\forall x\, (x\in A \iff (x = u \lor x = v))\]

    En pocas palabras, se garatiza que dados dos conjuntos $u$ y $v$, existe el conjunto 
    $\{u, v\}$.
    
    \textbf{Teorema 5:} El conjunto garantizado por el axioma 4 es único.
  \end{frame}

  \begin{frame}{Axioma del par}
    \[\forall u\, \forall v\, \exists A \,\forall x\, (x\in A \iff (x = u \lor x = v))\]

    En pocas palabras, se garatiza que dados dos conjuntos $u$ y $v$, existe el conjunto 
    $\{u, v\}$.
    
    \textbf{Teorema 5:} El conjunto garantizado por el axioma 4 es único.
    \textbf{Teorema 6} (referente a la redundancia): Sea $A$ un conjunto, entonces $\{A, A\} = \{A\}$.
  \end{frame}

  \begin{frame}{Axioma del par}
    \[\forall u\, \forall v\, \exists A \,\forall x\, (x\in A \iff (x = u \lor x = v))\]

    En pocas palabras, se garatiza que dados dos conjuntos $u$ y $v$, existe el conjunto 
    $\{u, v\}$.
    
    \textbf{Teorema 5:} El conjunto garantizado por el axioma 4 es único.
    \textbf{Teorema 6} (referente a la redundancia): Sea $A$ un conjunto, entonces $\{A, A\} = \{A\}$.

    Ahora si se puede garantizar la existencia de conjuntos distintos al vacío.

    \[\{\emptyset\}, \{\{\emptyset\}\}, \{\{\{\emptyset\}\}\}, \cdots\]
  \end{frame}

  \begin{frame}{Axioma de la unión}
    \[\forall F\, \exists U\, \forall x (x \in U \iff (\exists C\in F)(x \in C))\]

    \textbf{Teorema 7:} El conjunto garatizado por el axioma 5 es único.
  \end{frame}

  \begin{frame}{Axioma de la unión}
    \[\forall F\, \exists U\, \forall x (x \in U \iff (\exists C\in F)(x \in C))\]

    \textbf{Teorema 7:} El conjunto garatizado por el axioma 5 es único.

    Si existen dos conjuntos, $U_1$ y $U_2$ que cumplen las condiciones del axioma, entonces 
    
    \begin{equation*}
      \begin{aligned}
        x \in U_1 &\iff (\exists C\in F)(x\in C) \\
        &\iff x \in U_2
      \end{aligned}
    \end{equation*}

    Al ser un conjunto que siempre existe y es único se denota como $\bigcup F$ y se dice que 
    $x\in \bigcup F \iff (\exists C\in F)(x\in C)$
  \end{frame}

  \begin{frame}{Unión de conjuntos}
    En las diapositivas anteriores se empezó a hablar de operaciones con conjuntos, la intersección y 
    diferencia. Para la tería también es importante construir conjuntos de manera menos restricitva, por eso 
    se usa el axioma 5, que garatiza el siguiente teorema.

    \textbf{Teorema 8:} Dados dos conjuntos $A$ y $B$ existe un único conjunto $C$ tal que 
    $x\in C \iff (x\in A \lor x\in B)$.
  \end{frame}

  \begin{frame}{Unión de conjuntos}
    \textbf{Teorema 8:} Dados dos conjuntos $A$ y $B$ existe un único conjunto $C$ tal que 
    $x\in C \iff (x\in A \lor x\in B)$.

    Dados $A, B$ conjuntos, entonces por el axioma del par, existe el conjunto $\{A,\, B\}$ y por el axioma de 
    la unión existe $\bigcup \{A,\, B\} := A \cup B$ (notación). Veamos que este conjunto cumple la condición
    del teorema.
  \end{frame}

  \begin{frame}{Unión de conjuntos}
    \textbf{Teorema 8:} Dados dos conjuntos $A$ y $B$ existe un único conjunto $C$ tal que 
    $x\in C \iff (x\in A \lor x\in B)$.

    Dados $A, B$ conjuntos, entonces por el axioma del par, existe el conjunto $\{A,\, B\}$ y por el axioma de 
    la unión existe $\bigcup \{A,\, B\} := A \cup B$ (notación). Veamos que este conjunto cumple la condición
    del teorema.

    \begin{itemize}
      \item $\Longrightarrow)$ 

            Si $x\in A\cup B$ entonces $x\in A \lor x\in B$, esto por el axioma de la unión.
    \end{itemize}
  \end{frame}
  

  \begin{frame}{Unión de conjuntos}
    \textbf{Teorema 8:} Dados dos conjuntos $A$ y $B$ existe un único conjunto $C$ tal que 
    $x\in C \iff (x\in A \lor x\in B)$.

    Dados $A, B$ conjuntos, entonces por el axioma del par, existe el conjunto $\{A,\, B\}$ y por el axioma de 
    la unión existe $\bigcup \{A,\, B\} := A \cup B$ (notación). Veamos que este conjunto cumple la condición
    del teorema.

    \begin{itemize}
      \item $\Longrightarrow)$ 

            Si $x\in A\cup B$ entonces $x\in A \lor x\in B$, esto por el axioma de la unión.

      \item $\Longleftarrow)$

            Si $x\in A \lor x\in B$, entonces $x \in \bigcup  \{A,\, B\} := A\cup B$.
    \end{itemize}
  \end{frame}

  \begin{frame}{Unión de conjuntos}
    \textbf{Teorema 8:} Dados dos conjuntos $A$ y $B$ existe un único conjunto $C$ tal que 
    $x\in C \iff (x\in A \lor x\in B)$.

    Dados $A, B$ conjuntos, entonces por el axioma del par, existe el conjunto $\{A,\, B\}$ y por el axioma de 
    la unión existe $\bigcup \{A,\, B\} := A \cup B$ (notación). Veamos que este conjunto cumple la condición
    del teorema.

    \begin{itemize}
      \item $\Longrightarrow)$ 

            Si $x\in A\cup B$ entonces $x\in A \lor x\in B$, esto por el axioma de la unión.

      \item $\Longleftarrow)$

            Si $x\in A \lor x\in B$, entonces $x \in \bigcup  \{A,\, B\} := A\cup B$.
    \end{itemize}

    En la misma demostración denotamos este conjunto único (¿?) como $A \cup B$.
  \end{frame}

  \begin{frame}{Ejercicio}
    \textbf{Teorema 9:} Sea $A\neq \emptyset$ y $A\subseteq B$. entonces $\bigcup A \subseteq \bigcup B$ y 
    $\bigcap B \subseteq \bigcap A$.
  \end{frame}

  \begin{frame}{Ejercicio}
    \textbf{Teorema 9:} Sea $A\neq \emptyset$ y $A\subseteq B$. entonces $\bigcup A \subseteq \bigcup B$ y 
    $\bigcap B \subseteq \bigcap A$.

    \begin{equation*}
      \begin{aligned}
        x \in \bigcup A &\iff (\exists D\in A)(x\in D) 
      \end{aligned}
    \end{equation*}
  \end{frame}

  \begin{frame}{Ejercicio}
    \textbf{Teorema 9:} Sea $A\neq \emptyset$ y $A\subseteq B$. entonces $\bigcup A \subseteq \bigcup B$ y 
    $\bigcap B \subseteq \bigcap A$.

    \begin{equation*}
      \begin{aligned}
        x \in \bigcup A &\iff (\exists D\in A)(x\in D) \\
        &\Rightarrow (\exists D\in B)(x\in D) 
      \end{aligned}
    \end{equation*}
  \end{frame}

  \begin{frame}{Ejercicio}
    \textbf{Teorema 9:} Sea $A\neq \emptyset$ y $A\subseteq B$. entonces $\bigcup A \subseteq \bigcup B$ y 
    $\bigcap B \subseteq \bigcap A$.

    \begin{equation*}
      \begin{aligned}
        x \in \bigcup A &\iff (\exists D\in A)(x\in D) \\
        &\Rightarrow (\exists D\in B)(x\in D) \\
        &\iff x \in \bigcup B
      \end{aligned}
    \end{equation*}
  \end{frame}

  \begin{frame}{Ejercicio}
    \textbf{Teorema 9:} Sea $A\neq \emptyset$ y $A\subseteq B$. entonces $\bigcup A \subseteq \bigcup B$ y 
    $\bigcap B \subseteq \bigcap A$.

    \begin{equation*}
      \begin{aligned}
        x \in \bigcap B &\iff (\forall D\in B)\,(x\in D)
      \end{aligned}
    \end{equation*}
  \end{frame}

  \begin{frame}{Ejercicio}
    \textbf{Teorema 9:} Sea $A\neq \emptyset$ y $A\subseteq B$. entonces $\bigcup A \subseteq \bigcup B$ y 
    $\bigcap B \subseteq \bigcap A$.

    \begin{equation*}
      \begin{aligned}
        x \in \bigcap B &\iff (\forall D\in B)\,(x\in D) \\
        &\Rightarrow (\forall D\in A)(x\in A)
      \end{aligned}
    \end{equation*}
  \end{frame}

  \begin{frame}{Ejercicio}
    \textbf{Teorema 9:} Sea $A\neq \emptyset$ y $A\subseteq B$. entonces $\bigcup A \subseteq \bigcup B$ y 
    $\bigcap B \subseteq \bigcap A$.

    \begin{equation*}
      \begin{aligned}
        x \in \bigcap B &\iff (\forall D\in B)\,(x\in D) \\
        &\Rightarrow (\forall D\in A)(x\in A) \\
        &\iff x\in \bigcap A
      \end{aligned}
    \end{equation*}
  \end{frame}

  \begin{frame}{Axioma de partes}
    \[\forall A\, \exists P\, \forall x\, (x\in P \iff x\subseteq A)\]

    \textbf{Teorema 10:} El conjunto garantizado por el axioma de partes es único.
  \end{frame}

  \begin{frame}{Axioma de partes}
    \[\forall A\, \exists P\, \forall x\, (x\in P \iff x\subseteq A)\]

    \textbf{Teorema 10:} El conjunto garantizado por el axioma de partes es único. 
    
    Si existen $P_1$ y $P_2$ que cumplen las condiciones del axioma, entonces
    \begin{equation*}
      \begin{aligned}
        x\in P_1 &\iff x\subseteq A
      \end{aligned}
    \end{equation*}
  \end{frame}

  \begin{frame}{Axioma de partes}
    \[\forall A\, \exists P\, \forall x\, (x\in P \iff x\subseteq A)\]

    \textbf{Teorema 10:} El conjunto garantizado por el axioma de partes es único. 
    
    Si existen $P_1$ y $P_2$ que cumplen las condiciones del axioma, entonces
    \begin{equation*}
      \begin{aligned}
        x\in P_1 &\iff x\subseteq A \\
        &\iff x\in P_2
      \end{aligned}
    \end{equation*}
  \end{frame}

  \begin{frame}{Axioma de partes}
    \[\forall A\, \exists P\, \forall x\, (x\in P \iff x\subseteq A)\]

    \textbf{Teorema 10:} El conjunto garantizado por el axioma de partes es único. 
    
    Si existen $P_1$ y $P_2$ que cumplen las condiciones del axioma, entonces
    \begin{equation*}
      \begin{aligned}
        x\in P_1 &\iff x\subseteq A \\
        &\iff x\in P_2
      \end{aligned}
    \end{equation*}

    De nuevo, por la existencia y la unicidad podemos darle un nombre, dado un conjunto $A$, 
    denotamos por $P(A)$.
  \end{frame}
\end{document}
