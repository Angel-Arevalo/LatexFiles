\documentclass{beamer}
\usepackage{mathrsfs}

\begin{document}
  \begin{frame}{Axiomas y teorías coherentes}
    
    \begin{center}
      {\Large ¿Qué es un axioma?}
    \end{center}

    Es un enunciado que se toma como válido sin la necesidad de demostrarlo.

    Son las semillas de las teorías, ya que a partir de esos supuestos se deducen proposiciones.
    Esto último nos indica que toda teoría axiomática es tautologica. 


    Una teoría coherente es una cantidad de axiomas que no se contradicen entre si y que además
    son independientes, es decir, un axioma no se puede deducir de otros.


    Cantor fué el precursor de la teoría de hoy en día, pero los axiomas que usó (que eran intuitivos)
    no formaban una teoría coherente, a continuación recalcamos por qué.
  \end{frame}

  \begin{frame}
    \textbf{Principio de comprensión:} Dada una propiedad $P(x)$, un conjunto es el formado por todos
    los elementos (de algún universo) que cumplen dicha propiedad.
    
    Esta fué la definición de conjunto usada en los principios. Tomemosla como cierta, es decir, consideremosla
    un axioma.

    En 1901, Russell considero el siguiente conjunto $\mathscr{R} = \{x: x\notin \mathscr{R}\}$. 

    Como consideramos el principio de comprensión verdadero, entonces $\mathscr{R}$ es un conjunto. Veamos que la existencia de 
    este conjunto es una contradicción. 
    \newline\newline

    Si $\mathscr{R}\in \mathscr{R}$, por la definición del conjunto de Russell, $\mathscr{R} \notin \mathscr{R}$, que 
    claramente es una contradicción lógica.
    \newline
    Como el primer caso es falso, analizamos cuando $\mathscr{R} \notin \mathscr{R}$, de lo que se deduce que 
    $\mathscr{R} \in \mathscr{R}$, llevando de nuevo a una contradicción.
    \newline
    
    Con esto último se atacó esta teoría, pues si se toma el principio de comprensión como verdadero, implica una proposición
    falsa.
  \end{frame}

  \begin{frame}{Axiomática de Zermelo-Fraenkel}
    Zermelo fue el que continuó con la formalización de las matemáticas y Fraenkel continuó su trabajo.
    \newline\newline
    Con el trabajo de muchos matemáticos se determinaron que con 9 axiomas se puede deducir toda la matemática
    trabajada hasta el momento. Antes que nada, cabe recalcar que la noción de conjunto es libre en la teoría, 
    es decir, que no se hace una definición de lo que es un conjunto ni mucho menos de $\in$, ya que es clave que 
    en la teoría de conjuntos todo es un conjunto, y ya que intuitivamente sabemos que los conjuntos son colecciones de objetos,
    los conjuntos coleccionan cojuntos, por eso mismo no se hace una definición precisa.
  \end{frame}

  \begin{frame}{Axiomática de Zermelo-Fraenkel}
    \begin{itemize}
      \item $\forall A \,\forall B\, (A = B \iff \forall x\, (x\in A \iff x \in B))$
      \item $\exists \emptyset \,\forall x\, (x\notin \emptyset)$
      \item $\forall  A\, \exists S \,\forall x\,(x\in S \iff (x\in A \land P(x))$
      \item $\forall u\, \forall v\, \exists A \,\forall x\, (x\in A \iff (x = u \lor x = v))$
      \item $\forall F\, \exists U\, \forall x (x \in U \iff \exists C (x \in C \land C\in F))$
      \item $\forall F \,\exists Q\, \forall x\, (x \in Q \iff \forall y\, (y\in x \rightarrow y\in F))$
      \item $\exists N\, (\emptyset \in N \land \forall x\,(x\in I \rightarrow x\cup \{x\} \in N))$
      \item Axioma de remplazo
      \item Axioma de regularidad
    \end{itemize}

    Dentro de esta axiomática se puede incluir el axioma de elección pero al ser tan controversial su uso, se le dedica 
    una discución mucho mas larga.
  \end{frame}
\end{document}
