\documentclass{beamer}
\usepackage{mathrsfs}
\usepackage{amssymb}

\begin{document}
  \begin{frame}{Axiomas y teorías coherentes}
    
    \begin{center}
      {\Large ¿Qué es un axioma?}
    \end{center}

    Es un enunciado que se toma como válido sin la necesidad de demostrarlo.

    Son las semillas de las teorías, ya que a partir de esos supuestos se deducen proposiciones.
    Esto último nos indica que toda teoría axiomática es tautologica. 


    Una teoría coherente es una cantidad de axiomas que no se contradicen entre si y que además
    son independientes, es decir, un axioma no se puede deducir de otros.


    Cantor fué el precursor de la teoría de hoy en día, pero los axiomas que usó (que eran intuitivos)
    no formaban una teoría coherente, a continuación recalcamos por qué.
  \end{frame}

  \begin{frame}
    \textbf{Principio de comprensión:} Dada una propiedad $P(x)$, un conjunto es el formado por todos
    los elementos (de algún universo) que cumplen dicha propiedad.
    
    Esta fué la definición de conjunto usada en los principios. Tomemosla como cierta, es decir, consideremosla
    un axioma.

    En 1901, Russell considero el siguiente conjunto $\mathscr{R} = \{x: x\notin \mathscr{R}\}$. 

    Como consideramos el principio de comprensión verdadero, entonces $\mathscr{R}$ es un conjunto. Veamos que la existencia de 
    este conjunto es una contradicción. 
    \newline\newline

    Si $\mathscr{R}\in \mathscr{R}$, por la definición del conjunto de Russell, $\mathscr{R} \notin \mathscr{R}$, que 
    claramente es una contradicción lógica.
    \newline
    Como el primer caso es falso, analizamos cuando $\mathscr{R} \notin \mathscr{R}$, de lo que se deduce que 
    $\mathscr{R} \in \mathscr{R}$, llevando de nuevo a una contradicción.
    \newline
    
    Con esto último se atacó esta teoría, pues si se toma el principio de comprensión como verdadero, implica una proposición
    falsa.
  \end{frame}

  \begin{frame}{Axiomática de Zermelo-Fraenkel}
    Zermelo fue el que continuó con la formalización de las matemáticas y Fraenkel continuó su trabajo.
    \newline\newline
    Con el trabajo de muchos matemáticos se determinaron que con 9 axiomas se puede deducir toda la matemática
    trabajada hasta el momento. Antes que nada, cabe recalcar que la noción de conjunto es libre en la teoría, 
    es decir, que no se hace una definición de lo que es un conjunto ni mucho menos de $\in$, ya que es clave que 
    en la teoría de conjuntos todo es un conjunto, y ya que intuitivamente sabemos que los conjuntos son colecciones de objetos,
    los conjuntos coleccionan cojuntos, por eso mismo no se hace una definición precisa.
  \end{frame}

  \begin{frame}{Axiomática de Zermelo-Fraenkel}
    \begin{itemize}
      \item $\forall A \,\forall B\, (A = B \iff \forall x\, (x\in A \iff x \in B))$
      \item $\exists \emptyset \,\forall x\, (x\notin \emptyset)$
      \item $\forall  A\, \exists S \,\forall x\,(x\in S \iff (x\in A \land P(x))$
      \item $\forall u\, \forall v\, \exists A \,\forall x\, (x\in A \iff (x = u \lor x = v))$
      \item $\forall F\, \exists U\, \forall x (x \in U \iff \exists C (x \in C \land C\in F))$
      \item $\forall F \,\exists Q\, \forall x\, (x \in Q \iff \forall y\, (y\in x \rightarrow y\in F))$
      \item $\exists N\, (\emptyset \in N \land \forall x\,(x\in I \rightarrow x\cup \{x\} \in N))$
      \item Axioma de remplazo
      \item Axioma de regularidad
    \end{itemize}

    Dentro de esta axiomática se puede incluir el axioma de elección pero al ser tan controversial su uso, se le dedica 
    una discución mucho mas larga.
  \end{frame}

  \begin{frame}{Axioma de extensión}
    \[\forall A \,\forall B\, (A = B \iff \forall x\, (x\in A \iff x \in B))\]
    En nuestro lenguaje, se dice que dos conjuntos son iguales si todos los elementos de un conjunto son elementos también del otro.

    Seguimos la notación $A \subseteq B \iff \forall x\, (x\in A \rightarrow x\in B)$ y se lee "$A$ es un subconjunto de $B$".

    Bajo esta última notación, el primer axioma, o axioma de extensión dice que $A = B \iff (A\subseteq B \land B\subseteq A)$.
  \end{frame}

  \begin{frame}{Axioma del conjunto vacío}
    \[\exists \emptyset \,\forall x\, (x\notin \emptyset)\]

    Vacicamente se nos dice que exite un conjunto sin elementos. El axioma no dice que sea único, entonces probemoslo, pero antes, un 
    hecho muy importante.

    \textbf{Teorema 1:} Sea $A$ un conjunto, entonces $\emptyset \subseteq A$.
    
    \textbf{Teorema 2:} Si existe un conjunto $A$ tal que para todo conjunto $x$, $x\notin A$, entonces $A = \emptyset$
    
    Las demostraciones de estos dos hechos se denominan vacuidad.
  \end{frame}

  \begin{frame}{Axioma del subconjunto}
    \[\forall  A\, \exists S \,\forall x\,(x\in S \iff (x\in A \land P(x))\]
    
    Si tenemos una propiedad $P(x)$, se dice que  el conjunto formado por los elementos de $A$ que cumplen la propiedad
    existe.
    
    La notación más corta y util a la vez resulta ser $S = \{x\in A: P(x)\} := x\in A \land P(x)$.

    De nuevo, el axioma no garantiza igualdad.

    \textbf{Teorema 3:} Sea $\varphi(x)$ una propiedad. Si existen $S_1, S_2$ tales que para todo conjunto $x$, 
    $(x\in S_1 \iff (x\in A \land \varphi(x)))$ y $(x\in S_2 \iff (x\in A \land \varphi(x)))$, entonces $S_1 = S_2$.

    \begin{equation*}
      \begin{aligned}
        x\in S_1 &\iff x\in A \land \varphi(x) \\
        &\iff x \in S_2
      \end{aligned}
    \end{equation*}

    Podemos definir dos operaciones conjuntistas con ayuda de este axioma, sean $A$ y $B$ conjuntos. Sea $\varphi(x) = x \in B$,
    Entonces existe el conjunto $A\cap B = \{x\in A: \varphi(x)\}$ y $A\setminus B = \{x\in A: \neg\varphi(x)\}$.
  \end{frame}

  \begin{frame}
    El primero conjunto es llamado intersección de $A$ con $B$ y el segundo, diferencia de $A$ con $B$.

    La forma en que se deinió la intersección en la presentación anterior tiene bastantes propiedades para mostrar, una de ellas, 
    la conmutatividad, además de que si $x \in A\cap B$, entonces $x\in A \land x\in B$. 
    
    Esto último nos permite usar las reglas lógicas demostradas en la primera presentación.

    \textbf{Ejercicio} Demuestre que $A \cap (B \setminus C) = (A \cap B)\setminus C$.

    \begin{equation*}
      \begin{aligned}
        x \in A \cap (B \setminus C) &\iff x\in A \land x\in(B\setminus C) \\
        &\iff x\in A \land (x \in B \land x \notin C) \\
        &\iff (x\in A \land x\in B) \land x\notin C \\
        &\iff x \in (A\cap B) \land x\notin C \\
        &\iff x\in (A \cap B)\setminus C
      \end{aligned}
    \end{equation*}
  \end{frame}

  \begin{frame}{Conjunto de todos los conjuntos}
    Como iremos viendo durante el curso, los axiomas no garantizan que existan conjuntos muy grandes, es más, no se garantiza 
    que se pueda hacer el principio de comprensión libremente. Al tratar de generalizar surge la duda sobre el conjunto de todos los 
    conjuntos, y aún más importante, ¿Es un conjunto?

    Supongamos que existe tal conjunto y llamemoslo $\mathscr{O}$, es decir, $\forall x\, (x\in \mathscr{O})$. Por el axioma del subconjunto 
    existe $\mathscr{R} = \{x\in\mathscr{O}: x\notin x\}$, ahora apliquemos un argumento similar al de Russell.
    
    \begin{itemize}
      \item $\mathscr{R} \in \mathscr{R} \iff \mathscr{R} \in \mathscr{O} \land \mathscr{R} \notin \mathscr{R}$
    \end{itemize}

    Sabemos que $\mathscr{R}\in \mathscr{O}$ es una tautología, entonces $\mathscr{R} \in \mathscr{O} \land \mathscr{R} \notin \mathscr{R} \iff \mathscr{R} \notin \mathscr{R}$,
    luego, $\mathscr{R} \in \mathscr{R} \iff \mathscr{R} \notin \mathscr{R}$, lo que claramente es una contradicción.

    Como la hipótesis de que $\mathscr{R} \in \mathscr{R}$ lleva a un absurdo, se concluye que $\mathscr{R} \notin \mathscr{R}$, si y sólo si 
    $\mathscr{R} \notin \mathscr{O} \lor \mathscr{R} \in \mathscr{R}$.
  \end{frame}
\end{document}
