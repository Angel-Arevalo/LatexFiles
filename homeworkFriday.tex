\documentclass{article}
\usepackage{amsmath}
\usepackage{tikz}
\usetikzlibrary{arrows.meta}
\usepackage[spanish]{babel}
\usepackage[a4paper, margin=2.54cm]{geometry}
\usepackage{setspace}
\doublespacing
\usepackage{times}

\begin{document}
    \begin{titlepage}
        \begin{center}
            % Título principal
            {\Huge \textbf{ACA FINAL}\\[8cm]

            % Nombres de los estudiantes
            {\Large \textbf{Estudiante}}\\[0.5cm]
            \large
            Brandon Esleider Galicia Escarraga\\[10cm]


            % Información institucional
            {\large Lógica y Pensamiento Matemático\\
             José Fernando Velandia Tacuma}
        \end{center}
    \end{titlepage}

    \section{Parte 1: Expresiones Algebráicas}

        \subsection{Punto 1)}
            Simplifique la siguiente expresión y exprese el resultado en su forma reducida:
                \[
                    (3x^2 + 2x - 5) + (4x^2-7x+3) 
                \]

            
            Primero, vamos a agrupar los términos que tienen la misma indeterminada:
                \[(3x^2 + 4x^2)  + (2x - 7x) + (3 - 5)\]

            Ahora sumamos los coeficientes de los términos semejantes y llegamos a la expresión más simple posible:
                \[7x^2 - 5x - 2\]

        \subsection{Punto 2)}
            Realice la multiplicación y simplifique:
                \[
                    (x+2)(x-3)
                \]

            Hacemos la multiplicación, combinando cada término de la primera expresión con los de la segunda expresión,
            quedaría algo como:
                \[
                   (x^2 - 3x) + (2x - 6)
                \]

            Ahora agrupamos los términos semejantes
                \[
                   x^2 + (2x - 3x) - 6
                \]

            Sumamos términos semejantes y llegamos la expresión más simple:
                \[x^2 - x - 6\]

    \section{Parte 2: Ecuaciones}
        \subsection{Punto 3)}
            Resuelva la siguiente ecuación lineal:
                \[5x - 7 = 18\]

            Sumamos un $7$ en ambos lados de la ecuación, obteniendo:
                \[5x = 18 + 7 = 25\]
            
            Multiplicamos por $\frac{1}{5}$ en ambos lados y obtenemos el valor de $x$:
                \[x = \frac{25}{5} = 5\]

        \subsection{Punto 4)}
            Resuelva la siguiente ecuación cuadrática mediante factorización:
                \[x^2 - 5x + 6 = 0\]

            Para completar nececitamos hallar dos números que sumados nos den $-5$ y que multiplicados resulten en $6$,
            los cuales resultan en $-2$ y $-3$, pues $-2-3 = -5$ y $(-2)\cdot(-3) = 6$, luego, 
            $(x - 3)(x - 2) = x^2 - 2x - 3x + 6 = x^2 - 5x + 6$, por consiguiente, logramos factorizar el cuadrado, ahora solo nececitamos
            resolver $(x - 3)(x - 2) = 0$, y vemos que esto se cumple cuando $x = 3$ o $x = 2$.

    \section{Parte 3: inecuaciones}
        \subsection{Punto 5)}
            Resuelva la inecuación y represente la solución gráficamente en la recta real
                \[2x + 3 > 7\]

            Restamos $3$ en ambos lados de la inecuación, llegando a 
                \[2x > 7 - 3= 4\]
            Multiplicando por $\frac{1}{2}$ obtenemos
                \[x > \frac{1}{2} \cdot 4 = 2\]

            A continuación, se hace la gráfica en la recta real
            \begin{center}
                \begin{tikzpicture}[>=Stealth, line cap=round]
                    \draw[<->] (-5,0)--(5,0);

                    \draw (0,0.1)--(0,-0.1) node[below] {$0$};
                    \draw (-1,0.1)--(-1,-0.1) node[below] {$-1$};
                    \draw (2,0.1)--(2,-0.1) node[below] at (2, -.1) {$2$};

                    \draw[->, ultra thick, red] (2.1,0)--(5,0);
                    \draw[red, line width=0.4pt] (2,0) circle (0.1cm); 
                \end{tikzpicture}
            \end{center}
            

        \subsection{Punto 6)}
            Resuelva la siguiente inecuación cuadrática
                \[x^2 - 4x \leq 5\]

            Sumamos $4$ en cada lado de la inecuación para poder utilizar la fórmula del trinomio cuadrado perfecto,
            $a^2 + 2ab + b^2 = (a+b)^2$, y obtenemos
                \[x^2 - 4x + 4 \leq 5 + 4 = 9\]
                \[(x-2)^2 \leq 9\]

            Sacamos raíz en ambos lados, pero tenemos en cuenta que $\sqrt{x^2} = \left|x\right|$, agrupamos
                \begin{align*}
                    \left|x - 2\right| \leq 3\\
                \end{align*}

            De esto último, sabemos que puede suceder que o $x-2 \ge 0$ o que $x - 2 < 0$.
            \begin{itemize}
                \item Si $x-2 \ge 0$: Se consigue que $x - 2 \leq 3$
                \item Si $x - 2 < 0$: Se consigue $-(x-2) \leq 3$, luego $x-2 \ge -3$
            \end{itemize}

            Con estos dos items se concluye la forma general para analizar intervalos
            \[
                -3 \leq x - 2 \leq 3
            \]

            Ya con esto es simple ver que lo único que debemos hacer para despejar la $x$ es sumar
            $2$ en toda la inecuación, y así encontrar la solución

            \[-1 \leq x \leq 5\]

            A continuación, se hace la gráfica en la recta real
            \begin{center}
                \begin{tikzpicture}[>=Stealth, line cap=round]
                    \draw[<->] (-5,0)--(7,0);

                    \draw (0,0.1)--(0,-0.1) node[below] {$0$};
                    \draw (-1,0.1)--(-1,-0.1) node[below] {$-1$};
                    \draw (5,0.1)--(5,-0.1) node[below] {$5$};

                    \draw[-, ultra thick, red] (-1,0)--(5,0);
                    \draw[fill=red, red, line width=0.4pt] (5,0) circle (0.1cm); 
                    \draw[fill=red, red, line width=0.4pt] (-1,0) circle (0.1cm); 
                \end{tikzpicture}
            \end{center}
\end{document}
