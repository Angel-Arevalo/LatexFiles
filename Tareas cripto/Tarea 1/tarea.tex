\documentclass{article}
\usepackage{amsmath}
\usepackage{amssymb}
\newcommand{\enc}[2]{\text{Enc}_{#1}(#2)}
\begin{document}
  
  \section{Punto 1)}
    Demuestre o refute la siguiente afirmación:


    Para todo esquema perfectamente secreto, para toda distribución sobre el espacio de mensajes $M$,
    para todo par $m, m'\in M$ y para todo $c \in C$ (con $P[C = c] > 0$), se cumple que 
    \[P[M =m| C = c] = P[M = m'| C = c]\]


    Consideremos el conjunto de mensajes $M = \{A, B\}$, el conjunto de llaves $K = \{0, 1\}$ y 
    el conjunto de cifrados $C = \{D, E\}$, con $P[K = 0] = P[K = 1] = \frac{1}{2}$ y 
      \[P[M = A] = \frac{2}{5} \quad \quad P[M = B] = \frac{3}{5}\]
    
    Consideremos la función de encriptado Enc:
      \begin{equation*}
        \begin{aligned}
          \enc{0}{A} = D \quad\quad \enc{1}{A} = E \\
          \enc{0}{B} = E \quad\quad \enc{1}{B} = D
        \end{aligned}
     \end{equation*}


    entonces $C$ se distribuye de la siguiente manera:

    \begin{equation*}
      \begin{aligned}
        P[C = D] &= P[C = D | M = A] P[M = A] + P[C = D| M =B]P[M = B] \\
                 &= P[K = 0] P[M = A] + P[K = 1] P[M = B] \\
                 &= \frac{1}{2} \cdot \frac{2}{5} + \frac{1}{2}\cdot \frac{3}{5}
                 = \frac{1}{2}\\
        P[C = E] &= P[C = E| M = A] P[M = A] + P[C = E | M = B] P[M = B] \\
                 &= P[K = 1] P[M = A] + P[K = 0] P[M = B] \\
                 &= \frac{1}{2} \cdot \frac{2}{5} + \frac{1}{2}\cdot \frac{3}{5} 
                 = \frac{1}{2}
      \end{aligned}
    \end{equation*}

    Veamos que este sistema es seguro, comprobando que $P[\enc{K}{A} = c] = P[\enc{K}{B} = c]$, para $c \in C$,
    lo haremos recordando que $P[\enc{K}{m} = c] = P[C = c| M = m]$.

    \begin{equation*}
      \begin{aligned}
        P[\enc{K}{A} = E] = P[C = E| M = A] = P[K = 1] = \frac{1}{2} \\
        P[\enc{K}{A} = D] = P[C = D| M = A] = P[K = 0] = \frac{1}{2} \\
        P[\enc{K}{B} = E] = P[C = E| M = B] = P[K = 0] = \frac{1}{2} \\
        P[\enc{K}{B} = D] = P[C = D| M = B] = P[K = 1] = \frac{1}{2}
      \end{aligned}
    \end{equation*}

    De esto último se concluye que es un sistema seguro.

    Ahora verifiquemos $P[M = B | C=E]$ y $P[M = A| C=E]$:

    \begin{equation*}
      \begin{aligned}
        P[M = B| C=E] &= P[C = E| M=B] \frac{P[M=B]}{P[C=E]} \\
                      &= \frac{1}{2} \frac{\frac{3}{5}}{\frac{1}{2}}= \frac{3}{5}\\
        P[M = A| C=E] &= P[C = E| M=A] \frac{P[M=A]}{P[C=E]} \\
                      &= \frac{1}{2} \frac{\frac{2}{5}}{\frac{1}{2}}= \frac{2}{5}
      \end{aligned}
    \end{equation*}

    De ahí se concluye que $P[\enc{K}{A} = c] \neq P[\enc{K}{B} = c]$.

  \section{Punto 3)}
    Sean $f, g: \mathbb{N} \to \mathbb{R}_{\ge 0}$ dos funciones negligibles tales que $f(n) \ge g(n)$ para todo $n \in \mathbb{N}$.
    Para cada una de las definiciones de $f'(n)$ determine si necesariamente $f'$ es negligible bajo las 
    hipótesis dadas. En cada caso, pruebe su afirmación o proporcione un contraejemplo que respeste las 
    condiciones sobre $f$ y $g$.

    \subsection{Punto a)}
      \[f'(n) := \frac{1}{2^{\log^2 n}}\]

      Sea $c \in \mathbb{N}$ y $n > N$ donde $N$ es el menor natural tal que $\log N > c$ y como 
      el $\log$ es una función creciente, entonces 
      \[\log n > c\]
      Entonces se cumplen las siguientes desigualdades

      \begin{equation*}
        \begin{aligned}
          c < \log n \\
          c \log n < \log^2 n \\
          \log n^c < \log^2 n \\
          2^{\log n^c} < 2^{\log^2 n} \\
          n^c < 2^{\log^2 n} \\
          \frac{1}{2^{\log^2 n}} < \frac{1}{n^c}
        \end{aligned}
      \end{equation*}

    \subsection{Punto b)}
      \[f'(n) := f(n) + g(n)\]

      Sea $c>0$ y consideremos $f'(n) := f(n) + g(n)$. Por hipótesis, $f$ y $g$ son negligibles, es decir
      \[\lim_{n \to \infty} f(n)n^c = \lim_{n \to \infty} g(n)n^c =0\]
      Entonces 
      
      \begin{equation*}
        \begin{aligned}
          \lim_{n \to \infty} f'(n)n^c &= \lim_{n \to \infty} (f(n) + g(n))n^c \\
          &= \lim_{n \to \infty} f(n)n^c + \lim_{n \to \infty} g(n)n^c \\
          &= 0 + 0 = 0
        \end{aligned}
      \end{equation*}

      Por lo tanto, se concluye que $f'$ es una función negigible.

    \subsection{Punto c)}
      \[f'(n) := \frac{g(n)}{f(n)}\]
      
      Sea $f(n) = \frac{1}{2^n}$ y $g(n) = \frac{1}{2^{n+1}}$, con estas funciones se cumplen las 
      condiciones, es decir, $f(n) > g(n)$ y son negligibles, es decir, 
      \[\lim_{n \to \infty} f(n)n^c = \lim_{n \to \infty} g(n)n^c = 0 \text{ para todo } c\in \mathbb{N}\]

      Con esto en mente, vemos que 
      \[f'(n) = \frac{\frac{1}{2^{n+1}}}{\frac{1}{2^n}} = \frac{2^n}{2^{n+1}} = \frac{1}{2}\]

      y como $\lim_{n \to \infty} f'(n) \neq 0$, entonces $f'$ no es negligible, concluyendo así que el cociente bajo 
      esas condiciones no es necesariamente negligible.
\end{document}
