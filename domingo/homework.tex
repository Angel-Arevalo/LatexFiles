\documentclass{article}
\usepackage{amsmath}
\usepackage{amssymb}
\usepackage{graphicx}
%\usepackage[spanish]{babel}
\usepackage[a4paper, margin=2.54cm]{geometry}
\usepackage{setspace}
\doublespacing
\usepackage{times}


\begin{document}
    \begin{titlepage}
        {}
        \begin{center}
            % Título principal
          {\Huge \textbf{\fontsize{30}{28}\selectfont ACA Final}}\\[8cm]

            % Nombres de los estudiantes
            {\Large \textbf{\fontsize{23}{28} \selectfont Estudiante}}\\[0.5cm]
            \large
            Brandon Esleider Galicia Escarraga\\[10cm]


            % Información institucional
            {\large Análisis Avanzado de Cálculo Diferencial\\
                    Robinson F. Verano P }
        \end{center}
    \end{titlepage}

    \tableofcontents
    \newpage

    \section{Ejercicio 1}
      Derivabilidad de una función a trozos y análisis de concavidad

      Sea $f$ la función definida poe tramos como sigue:
        \begin{equation}
        f(x) =
        \begin{cases}
            a \sin\!\left(\dfrac{\pi x}{2}\right) + b, & \text{si } x \leq 1,\\[4pt]
            \dfrac{x^2 - 1}{1 - \sqrt{x}},            & \text{si } x > 1.
          \end{cases}
        \end{equation}

    \subsection{Punto a)}
      Determine los valores de $a$ y de $b$ para que $f$ sea continua y derivable en $x = 1$.

      Determinamos las derivadas laterales al rededor de $x = 1$, empezamos con la derivada 
      por derecha

    \begin{equation}
      \begin{aligned}
        f'_+(x) &= \frac{2x(1-\sqrt{x}) - (x^2 - 1)\left(\frac{-1}{2\sqrt{x}}\right)}{(1-\sqrt{x})^2} \\[4pt]
          &= \frac{1}{2\sqrt{x}} \cdot \frac{4x^{\frac{3}{2}} (1-\sqrt{x}) + x^2 - 1}{(1-\sqrt{x})^2} \\[4pt]
          &= \frac{1}{2\sqrt{x}} \frac{4x^{\frac{3}{2}} -4x^2 + x^2 - 1}{(1-\sqrt{x})^2} \\[4pt]
          &= \frac{1}{2\sqrt{x}} \frac{4x^{\frac{3}{2}} - 3x^2 - 1}{(1-\sqrt{x})^2}
      \end{aligned}
    \end{equation}
      Ahora hallamos la derivada por izquierda
    \begin{equation}
      \begin{aligned}
        f'_-(x) &= a\cos(\frac{\pi x}{2}) \frac{\pi}{2}
      \end{aligned}
    \end{equation}

    Independientemente del valor de $a$ o de $b$, $f'_-(1) = 0$ y para el valor de $f'_+(1)$ debemos tener en cuenta
    $\lim_{x\to 1^+} f'_+(x) = \lim_{x\to 1^+} \frac{1}{2\sqrt{x}} \frac{4x^{\frac{3}{2}} - 3x^2 - 1}{(1-\sqrt{x})^2} = 
    \frac{1}{2}\lim_{x\to 1^+} \frac{4x^{\frac{3}{2}} - 3x^2 - 1}{(1-\sqrt{x})^2} $. Como tanto numerdor y denominador, van para $0$,
    por lo tanto podemos aplicar la regla de L'Hopital, por lo tanto, obtenemos que 

    \begin{equation}
      \begin{aligned}
      \frac{1}{2}\lim_{x\to 1^+} \frac{4x^{\frac{3}{2}} - 3x^2 - 1}{(1-\sqrt{x})^2} = 
      \frac{1}{2}\lim_{x\to 1^+} \frac{6 x^{\frac{1}{2}} - 6x}{2(1-\sqrt{x})\frac{-1}{2\sqrt{x}}} \\[4pt] = 
      \frac{1}{2}\lim_{x\to 1^+} -\sqrt{x} \frac{6 x^{\frac{1}{2}} - 6x}{1-\sqrt{x}} = \frac{-1}{2}\lim_{x\to 1^+} \frac{6 x^{\frac{1}{2}} - 6x}{1-\sqrt{x}}
      \end{aligned}
    \end{equation}

    De nuevo, en este límite podemos ver que tanto numerador como denominador van para $0$, concluyendo así que podemos aplicar
    la regla de L'Hopital de nuevo, obteniendo el límite 
    \begin{equation}
      \begin{aligned}
      \frac{-1}{2}\lim_{x\to 1^+} \frac{6 x^{\frac{1}{2}} - 6x}{1-\sqrt{x}} = \frac{-1}{2}\lim_{x\to 1^+} \frac{\frac{3}{\sqrt{x}} - 6}{\frac{-1}{2\sqrt{x}}} 
      = \frac{-1}{2}(3 - 6)(-2) = \frac{-1}{2}6 = -3
      \end{aligned}
    \end{equation}

    Con esto en mente, como las derivadas laterales no coinciden, la función no es derivable para ningún $a, b \in \mathbb{R}$


    Para analizar la continuidad, vemos que $f(1) = a \sin(\frac{\pi}{2}) + b = a + b$, pero, $\lim_{x\to 1^+} f(x) = \lim_{x\to 1^+} \frac{x^2 - 1}{1-\sqrt{x}}$,
    y como tanto  numerador y denominador van para 0 cuando $x \to 1^+$, entonces podemos aplicar de nuevo la regla de L'Hopital y así 
    \begin{equation}
      \begin{aligned}
      \lim_{x\to 1^+} \frac{x^2 - 1}{1-\sqrt{x}} = \lim_{x\to 1^+} \frac{2x}{\frac{-1}{2\sqrt{x}}} = -4
      \end{aligned}
    \end{equation}

    Por lo tanto, $a + b = -4$ para que $f$ sea continua en $x = 1$.


    \subsection{Punto b)}
      Con los valores encontrados, analice la concavidad de $f$ en el intervalo $(1, \infty)$, calcule $f'(x), f''(x)$, identifique puntos de 
      inflexión y describa la concavidad por intervalos.

      En el intervalo dado, 
      \begin{equation}
        \begin{aligned}
          f(x) &= \frac{x^2 - 1}{1 - \sqrt{x}} = \frac{(x-1)(x+1)}{1 - \sqrt(x)} \\[4pt]
            &=- \frac{(1 - x)(1+x)}{1 - \sqrt{x}} \\[4pt]
            &= - \frac{(1-\sqrt{x})(1+\sqrt{x})(1+x)}{1-\sqrt{x}} \\[4pt] 
            &= -(1 + \sqrt{x})(1+x) = -(1 + x + \sqrt{x} + x^{\frac{3}{2}})
        \end{aligned}
      \end{equation}
      esta es una 
      expresión mucho más fácil de trabajar de ahora en adelante. Hallemos las dos primeras derivadas
    
      \begin{equation}
        \begin{aligned}
          f'(x) = -(1 + \frac{1}{2\sqrt{x}} + \frac{3}{2} \sqrt{x}) \\[4pt]
          f''(x) = -(\frac{-1}{4x^{\frac{3}{2}}} + \frac{3}{4\sqrt{x}})
        \end{aligned}
      \end{equation}
      

      Hallemos los puntos de inflexión, para ello, resolvemos $f''(x) = 0$, es decir, $\frac{-1}{4x^{\frac{3}{2}}} + \frac{3}{4\sqrt{x}} = 0$, entonces
      $\frac{3}{4\sqrt{x}} = \frac{1}{4x^{\frac{3}{2}}}$, es decir, $3x^{\frac{3}{2}} = \sqrt{x}$, por lo tanto $x = \frac{1}{3}$, pero no está en el 
      intervalo que estamos análizando, conclimos que no hay puntos de inflexión en el intervalo $(1, \infty)$. 

      Ahora, para análizar la concavidad de $f$, determinamos los signos de $f''$, pero del anterior párrafo, la segunda derivada no cambia de 
      signo. Fijemonos en la siguiente desigualdad, como $1 < x$, entonces $\frac{1}{3} < x$, luego
      \begin{equation}
        \begin{aligned}
          1 &< 3x, \\[4pt]
          \sqrt{x} &< 3x^{\frac{3}{2}}, \\[4pt]
          \frac{1}{4} \sqrt{x} &< \frac{1}{4}\, 3x^{\frac{3}{2}}, \\[4pt]
          \frac{1}{4x^{\frac{3}{2}}} &< \frac{3}{4\sqrt{x}}
          \;\Longrightarrow\;
          0 < -\frac{1}{4x^{\frac{3}{2}}} + \frac{3}{4\sqrt{x}}.
        \end{aligned}
      \end{equation}

      Esto último nos permite concluir que para todo $x > 1$, $f''(x) < 0$, es decir, que $f$ es concava hacia abajo en el intervalo $(1, \infty)$.

    \subsection{Punto c)}
      Basqueje la gráfica de $f$ mostrando el comportamiento de $f$ al rededor de $x \ge 1$. Indique los valores de continuidad y 
      derivabilidad encontrados.
      

      Del punto a sabemos que $a + b = -4$, es decir, $b = -4 -a$, y así $f$ quedaría definida como 
      \begin{equation}
        f(x) = 
        \begin{cases}
          a (\sin(\frac{\pi x}{2}) - 1) - 4, \text{ si } x \leq 1, \\[4pt]
          \frac{x^2 - 1}{1-\sqrt{x}}, \text{ si } x>1
        \end{cases}
      \end{equation}

      A continuación mostramos el bosquejo de la gráfica con un valor particular de $a$:
      \begin{figure}[h!]
        \begin{center}
          \includegraphics[width=0.75\textwidth]{images/geogebra-export.png}
        \end{center}
        \caption{Gráfica de la función $f$ definida con $a = 1$}\label{fig:}
      \end{figure}
      
      Para este valor de $a$, es fácil ver que $f$ es continua en $x = 1$, pero la forma en punta de la gráfica es la que 
      daña el hecho de que $f$ sea derivable porque no es suficientemente "suave". Además, se observa el hecho de que
      sea concava en el intervalo $(1, \infty)$.


  \section{Ejercicio 2}
      Considere la curva 
      \begin{equation}
        y^4 + x^3 = 2xy^2
      \end{equation}
      Derivamos implicitamente considerando $y := y(x)$

      \begin{equation}
        \begin{aligned}
          4y^3 y' + 3x^2 &= 2(y^2 + 2xy y') \\[4pt]
          4y^3 y' - 4xyy' &= 2y^2 - 3x^2 \\[4pt]
          y' &= \frac{2y^2 - 3x^2}{4y^3 - 4xy}
        \end{aligned}
      \end{equation}

      \subsection{Puntos de recta tangente horizontal}
      Hacemos $y' = 0$, es decir $\frac{2y^2 - 3x^2}{4y^3 - 4xy} = 0 \iff 2y^2 - 3x^2=0$, concluyendo así que $y^2 = \frac{3}{2}x^2$, luego,
      reemplazando en la ecuación de la curva, por un lado obtenemos

      \begin{equation}
        \begin{aligned}
          (\frac{3}{2}x^2)^2 + x^3 &= 2x(\frac{3}{2}x^2) \\[4pt]
          \frac{9}{4} x^4 + x^3 &= 3x^3 \\[4pt]
          \frac{9}{4} x^4 &= 2x^3 \\[4pt]
          9x = 8 &\iff x = \frac{8}{9}
        \end{aligned}
      \end{equation}

      Ahora, para hallar $y$ usamos la identidad hallada, $y^2 = \frac{3}{2} \cdot\frac{64}{81} = \frac{32}{27}$, luego 
      $y = \pm \frac{4\sqrt{2}}{3\sqrt{3}} = \pm\frac{4}{9} \sqrt{6}$. Falta hacer el análisi de que $4y^3 - 4xy \neq 0$, 
      pero $4y(y^2 - x) = 4y(\frac{32}{27} - \frac{8}{9}) \neq 0$, para ambos valores de $y$.

      Los puntos de rectas horizontales no triviales son $(\frac{8}{9}, \frac{4}{9} \sqrt{6})$ y $(\frac{8}{9}, -\frac{4}{9} \sqrt{6})$.

      \subsection{Puntos con recta tangente vertical}
          Análizamos $x' = 1/y' = 0$, es decir, $\frac{4y^3 - 4xy}{2y^2 - 3x^2} = 0$, es decir, $4y^3 = 4xy$ y $2y^2 \neq 3x^2$.
          De la primera ecuación $y^2 = x$, reemplazando en la ecuación de la curva, $x^2 + x^3 = 2x^2 \rightarrow x^3 = x^2$, entonces
          $x = 0$ o $x = 1$. Si $x = 0$, $y = 0$, y no se cumple que  $2y^2 \neq 3x^2$ y si $x = 1$, $y = \pm 1$, si $y = -1$, 
          $4(-1)^3 = 4(1)(-1)$ y si $y = 1$, $4(1)^3 = 4(1)(1)$, y para cualquier valor de $y$ que sea de los anteriormente
          mencionados, $2 \neq 3$, es decir, en los puntos $(1, \pm 1)$ hay una tangente vertical. El enunciado tiene un error.

  \section{Ejercicio 5}
    Use la definición de derivada para hallar la derivada de $f(x) = x^{\frac{-1}{3}}$ en $x = 8$.

    Empezamos usando la definición, y recordando que $a^3 - b^3 = (a-b)(a^2 + ab + b^2)$:

    \begin{equation}
      \begin{aligned}
        \lim_{x \to 8} \frac{x^\frac{-1}{3} - \frac{1}{2}}{x - 8} 
        &= \lim_{x\to 8}  x^\frac{-1}{3}\frac{1}{2} \cdot\frac{2 - x^\frac{1}{3}}{x-8} \\
        &= \frac{-1}{2}\lim_{x\to 8}  x^\frac{-1}{3} \lim_{x\to 8} \frac{x^\frac{1}{3}-2}{x-8} \\
        &= \frac{-1}{4} \lim_{x\to 8} \frac{x^\frac{1}{3}-2}{(x^\frac{1}{3}-2)(x^\frac{2}{3} + 2x^\frac{1}{3} + 4)} \\
        &= \frac{-1}{4} \lim_{x\to 8} \frac{1}{x^\frac{2}{3} + 2x^\frac{1}{3} + 4} \\
        &= \frac{-1}{4} \cdot\frac{1}{12} = \frac{-1}{48}
      \end{aligned} 
    \end{equation}


  \section{Ejercicio 6}
    Un contenedor cerrado está formado por un cilindro con tapas semiesféricas. El volumen total del contenedor es $V_0$ (constante). Determine el radio r y la altura
    h de la parte cilíndrica que minimizan el costo total del material, si el material de las tapas semiesféricas cuesta el doble que el material de los lados
    cilíndricos.


    Primero que todo, el volumen de este sólido viene dado por $V_0 = \pi r^2 h + \frac{4}{3} \pi r^3$, y el precio es proporcial a 
    $P(r, h) = 2\pi r h + 8\pi r^2$, ya que $2\pi r h$ determina el área superficial de el cilindro y $8\pi r^2 = 2\cdot 4\pi r^2$ es dos veces 
    el área superficial de las dos tapas, que unidas resultan en una esfera. 

    De la primera ecuación vamos a depejar un término similar a $2\pi h r$, vemos el proceso:

    \begin{equation}
      \begin{aligned}
        V_0 &= \pi r^2 h + \frac{4}{3} \pi r^3 \\
        V_0 - \frac{4}{3} \pi r^3  &= \pi r^2 h \\
        \frac{V_0}{r} - \frac{4}{3} \pi r^2  &= \pi r h \\
        \frac{2V_0}{r} - \frac{8}{3} \pi r^2  &= 2\pi r h 
      \end{aligned}
    \end{equation}

    Remmplazando esto en la ecuación $P$:

    \begin{equation}
      \begin{aligned}
        2\pi r h + 8\pi r^2 &= \frac{2V_0}{r} - \frac{8}{3} \pi r^2 + 8\pi r^2 \\
        &= \frac{2V_0}{r} + \frac{16}{3}\pi r^2 = H(r)
      \end{aligned}
    \end{equation}

    Ahora optimicemos $H$, primero hallando la derivada, que es lo que hacemos en lo que sigue 
    \begin{equation}
      \begin{aligned}
        H'(r) = - \frac{2V_0}{r^2} + \frac{32}{3} \pi r
      \end{aligned}
    \end{equation}
     
     Y como debemos minizar, reolvemos  $H'(r) = 0$:

    \begin{equation}
      \begin{aligned}
        - \frac{2V_0}{r^2} + \frac{32}{3} \pi r &= 0 \\
          \frac{32}{3} \pi r &= \frac{2V_0}{r^2} \\
          r^3 &= \frac{3V_0}{16\pi} \\
          r &= \sqrt[3]{\frac{3V_0}{16\pi}}
      \end{aligned}
    \end{equation}

    Hallamos $H''(r)$:
    \begin{equation}
      H''(r) = \frac{4V_0}{r^3} + \frac{32}{3} \pi
    \end{equation}

    Y como $H''(\sqrt[3]{\frac{3V_0}{16\pi}}) > 0$, porque es suma de positivos, entonces $r = \sqrt[3]{\frac{3V_0}{16\pi}}$ es un mínimo.
    

    De la ecuación $V_0 = \pi r^2 h + \frac{4}{3} \pi r^3$, dividimos por $\pi r^3$ y obtenemos $\frac{V_0}{\pi r^3} = \frac{h}{r} + \frac{4}{3}$,
    pero como $r^3 = \frac{3V_0}{16\pi}$, entonces queda $\frac{16}{3} = \frac{h}{r} + \frac{4}{3}$, entonces 
    $\frac{12}{3} = \frac{h}{r}$, luego $h = 4r = 4 \sqrt[3]{\frac{3V_0}{16\pi}}$.

  \section{Ejercicio 10}
    Demuestre que la función $f(x) = x^5 + 3x - 1$ tiene una raíz única en el intervalo $[0, 1]$.

    Primero que todo, sabemos que todo polinomio es una función continua, por lo tanto, se asumen todos los valores
    entre las imagenes de los extremos. Como $f(0) = -1$ y $f(1) = 3$, entonces todo valor $f(x) \in [-1, 3]$ si 
    $x \in [0, 1]$. Por ello, podemos afirmar que existe $x_0$ tal que $f(x_0) = 0$. 

    Supongamos que existe otro $x_1 \in [0, 1]$ tal que $f(x_1) = 0$. Sin pérdida de generalidad, asumamos que $x_1 > x_0$.
    Como $f$ es continua y diferenciable en $[x_0, x_1]$, por el teorema de Rolle, existe $x_2\in (x_0, x_1)$ tal que
    $f'(x_2) = 0$, es decir, $5x_2^4 + 3 = 0$, por lo tanto $x_2^4 = \frac{-3}{5}$, la cual es una clara contradiccón
    porque está diciendo que un número positivo es igual que uno negativo, esto no tiene solución en $\mathbb{R}$, concluyendo
    así que la única raíz es $x_0$.
\end{document}
